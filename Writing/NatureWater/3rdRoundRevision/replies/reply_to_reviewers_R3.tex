% Options for packages loaded elsewhere
\PassOptionsToPackage{unicode}{hyperref}
\PassOptionsToPackage{hyphens}{url}
%
\documentclass[
]{article}
\usepackage{lmodern}
\usepackage{amsmath}
\usepackage{ifxetex,ifluatex}
\ifnum 0\ifxetex 1\fi\ifluatex 1\fi=0 % if pdftex
  \usepackage[T1]{fontenc}
  \usepackage[utf8]{inputenc}
  \usepackage{textcomp} % provide euro and other symbols
  \usepackage{amssymb}
\else % if luatex or xetex
  \usepackage{unicode-math}
  \defaultfontfeatures{Scale=MatchLowercase}
  \defaultfontfeatures[\rmfamily]{Ligatures=TeX,Scale=1}
\fi
% Use upquote if available, for straight quotes in verbatim environments
\IfFileExists{upquote.sty}{\usepackage{upquote}}{}
\IfFileExists{microtype.sty}{% use microtype if available
  \usepackage[]{microtype}
  \UseMicrotypeSet[protrusion]{basicmath} % disable protrusion for tt fonts
}{}
\makeatletter
\@ifundefined{KOMAClassName}{% if non-KOMA class
  \IfFileExists{parskip.sty}{%
    \usepackage{parskip}
  }{% else
    \setlength{\parindent}{0pt}
    \setlength{\parskip}{6pt plus 2pt minus 1pt}}
}{% if KOMA class
  \KOMAoptions{parskip=half}}
\makeatother
\usepackage{xcolor}
\IfFileExists{xurl.sty}{\usepackage{xurl}}{} % add URL line breaks if available
\IfFileExists{bookmark.sty}{\usepackage{bookmark}}{\usepackage{hyperref}}
\hypersetup{
  hidelinks,
  pdfcreator={LaTeX via pandoc}}
\urlstyle{same} % disable monospaced font for URLs
\usepackage[margin=1in]{geometry}
\usepackage{graphicx}
\makeatletter
\def\maxwidth{\ifdim\Gin@nat@width>\linewidth\linewidth\else\Gin@nat@width\fi}
\def\maxheight{\ifdim\Gin@nat@height>\textheight\textheight\else\Gin@nat@height\fi}
\makeatother
% Scale images if necessary, so that they will not overflow the page
% margins by default, and it is still possible to overwrite the defaults
% using explicit options in \includegraphics[width, height, ...]{}
\setkeys{Gin}{width=\maxwidth,height=\maxheight,keepaspectratio}
% Set default figure placement to htbp
\makeatletter
\def\fps@figure{htbp}
\makeatother
\setlength{\emergencystretch}{3em} % prevent overfull lines
\providecommand{\tightlist}{%
  \setlength{\itemsep}{0pt}\setlength{\parskip}{0pt}}
\setcounter{secnumdepth}{-\maxdimen} % remove section numbering
\ifluatex
  \usepackage{selnolig}  % disable illegal ligatures
\fi

\author{}
\date{\vspace{-2.5em}}

\begin{document}

\hypertarget{reply-to-reviewer-1}{%
\section{Reply to Reviewer 1}\label{reply-to-reviewer-1}}

I have read the second revision of this manuscript and the authors'
responses to my last review. I think the authors have done an excellent
job in responding to the points I raised in that review. Thus, I
recommend acceptance after the authors address the very minor points
that I noted while reading this version of the manuscript. Those points
are as follows:

\textcolor{blue}{Thank you for your feedback on our manuscript. We have addressed all your comments and suggestions, and our replies to your comments can be found in blue below (original comments in black).}

\begin{itemize}
\tightlist
\item
  Footnote 1 -- change ``saturated thickness'' to ``aquifer thickness''
  to be consistent with usage elsewhere. Same for x-axis label on Figure
  C.2.
\end{itemize}

\textcolor{blue}{Thank you for catching this. In the footnote, "saturated thickness" was replaced with "aquifer thickness".}

\begin{itemize}
\tightlist
\item
  Line 110 -- can you briefly describe how this ``accounting'' is done?
\end{itemize}

\textcolor{blue}{The figure is designed to showcase the effect of aquifer thickness on the percentage of irrigated acres, without the influence of spatio-temporal variations in weather conditions and drought risk exposure across the aquifer. In deriving the relationship between aquifer thickness and the proportion of irrigated acres using our model, these factors are kept constant, allowing us to focus solely on the impact of aquifer thickness. We have opted to omit the latter part of the sentence to avoid potential confusion. While Figure 1 was constructed using a similar procedure—adjusting only the water deficit while keeping other factors unchanged — we did not explicitly state this either. We believe this omission avoids unnecessary confusion.}

\begin{itemize}
\tightlist
\item
  Line 147 -- ``leading in'' should be ``leading to''.
\end{itemize}

\textcolor{blue}{Thank you for catching this. In the footnote, "leading in" was replaced with "leading to".}

\begin{itemize}
\tightlist
\item
  Lines 282-283 -- delete ``of the aquifer'' - redundant.
\end{itemize}

\textcolor{blue}{Thank you for catching this. In the sentence, "of the aquifer" was removed to avoid redundancy.}

\begin{itemize}
\tightlist
\item
  Line 289 -- ``thickness'' should be ``thicknesses''.
\end{itemize}

\textcolor{blue}{Thank you for catching this. In the sentence, "thickness" was replaced with "thicknesses".}

\begin{itemize}
\tightlist
\item
  Figure 7 -- need a 200 m label for scale bar on soybean map. Are the
  blank areas in the aquifer in Texas and Oklahoma an indication that
  the aquifer is thin or that there is a lack of data? Please clarify.
\end{itemize}

\textcolor{blue}{The 200m label has been added to the legend of the soybean map. The blank area are counties that were not used in either of the yield and irrigation share regressions. This because neither of them have reported information on corn and/or soybean production from USDA, due to the small areas of land dvoted to production of these crops in these locations. 
}

\begin{itemize}
\tightlist
\item
  Line 308 -- ``y'' should be ``Y''.
\end{itemize}

\textcolor{blue}{Thank you for catching this. The suggested change has been made.}

\begin{itemize}
\tightlist
\item
  Lines 357-359 -- delete sentence as it is repeated on lines 363-364
  where it fits better.
\end{itemize}

\textcolor{blue}{Thank you for catching this. The sentenced was removed.}

\begin{itemize}
\tightlist
\item
  Lines 401-402 -- a recent paper in Agricultural Water Management
  addresses this issue in more detail
  (\url{https://www.sciencedirect.com/science/article/pii/S0378377423002731}).
\end{itemize}

\textcolor{blue}{Thank you for suggesting this article. We now cite this article at the end of the sentence.}

\begin{itemize}
\tightlist
\item
  Lines 410-413 -- very awkward sentence -- please rewrite.
\end{itemize}

\textcolor{blue}{We have rewritten the sentence to now state the following, which we feel improves the clarity of the text:}

\textcolor{blue}{"As discussed in section 1 of our manuscript, there is extensive evidence pointing to reduced well yields in these areas of the HPA. This evidence ranges from anecdotal and theoretical insights to simulation-based studies. Collectively, they suggest that these restrictions significantly constrain farmers' ability to meet crop water demands during droughts."}

\begin{itemize}
\tightlist
\item
  Line 446 -- delete ``across''.
\end{itemize}

\textcolor{blue}{Thank you for catching this. The suggested change has been made.}

\hypertarget{reply-to-reviewer-3}{%
\section{Reply to Reviewer 3}\label{reply-to-reviewer-3}}

The second round comments have been addressed convincingly. The
discussion of the role of groundwater management policies in Appendix A
is particularly useful addition, as it clarifies how the results
presented here relate to the findings of analyses which focused more on
policy factors. Ruling out regulations as a substantial causal factor
based on the limited temporal overlap with the study period is coherent.
The other edits implemented in response to the other comments have
further improved the strength and accuracy of the presentation.

Two relatively minor points remain to be addressed, based on one of the
responses and the data and code correction you mentioned:

\textcolor{blue}{Thank you for your feedback on our manuscript. We have addressed all your comments and suggestions, and our replies to your comments can be found in blue below (original comments in black).}

\begin{itemize}
\item
  \begin{enumerate}
  \def\labelenumi{(\arabic{enumi})}
  \tightlist
  \item
    I would recommend a minor addition regarding the response to my
    second round comment 5.b. You have pointed out that the claim of a
    significant yield decline in lines 77-78 is supported by Figure A.1
    (and also A.2 for soybeans). Please add a reference to both figures
    in this sentence to clarify for readers where they can find evidence
    for the statistical significance.
  \end{enumerate}
\end{itemize}

\textcolor{blue}{Thank you for this suggestion. We have modified the sentence as follows:}

\textcolor{blue}{As drought severity increases (i.e., as water deficit increases), rainfed yields of both crops decline significantly whereas irrigated yields are increasing or stable as deficit increases (see Figures A.1 and A.2 for the 95\% confidence intervals)}

\begin{itemize}
\item
  \begin{enumerate}
  \def\labelenumi{(\arabic{enumi})}
  \setcounter{enumi}{1}
  \tightlist
  \item
    The beginning of the response letter describes a code and data
    correction unrelated to the previous reviewer comments.
    ``{[}Counties{]} with only irrigated acres'' were removed from the
    irrigation share regression. The immediate acknowledgement of this
    correction is appreciated. I trust that it was based on sensible
    considerations, though the rationale does not immediately become
    clear from the short statement, especially given that this does not
    seem to have been relevant previously. As you point out, the result
    changes due to this correction are quantitatively relatively minor
    (most visible in Figs. 3 and 4), and do not affect the qualitative
    findings or conclusions.
  \end{enumerate}
\end{itemize}

\textcolor{blue}{We apologize for the confusion we caused. Our description of this issue was not clear in the last round of revision. We meant to highlight that we excluded counties where only the irrigated acres are "observed" because only irrigated production data were reported by USDA - typically this is because the total number of rainfed acres is small or was not captured by USDA surveys in the county. In these cases, we cannot  calculate the share of irrigated acres for those specific counties. Previously before fixing the coding error, those counties were assigned the value of 1 (100\%) for their share of irrigated acres because no rainfed production data was repoted by USDA, which may lead to an over-estimation of their actual share of irrigated acres. Therefore, in our revisions we removed these counties to avoid this issue and the consequent reduction of data quality. However, as previously, this did not alter the findings of our analysis, and therefore this is largely a cosmetic change to ensure we base our analysis on the highest quality data available.}

I recommend, however, making sure that the correction, its rationale,
and potential implications for the result interpretation are made clear
to readers. Lines 347-348 state that ``irrigated area share and average
crop yield estimates use data only from counties that have both
irrigated and rainfed production observations''. This statement covers
the removal of fully irrigated counties, but could be clearer: Lines
348-350 mention the removal of ``solely rainfed'' counties explicitly -
mentioning this but not the removal of fully irrigated counties could be
confusing to readers. Please include an explicit statement about the
removal of fully irrigated counties to avoid confusion. Please also make
clear whether this removal pertains to both regressions (as implied in
lines 347) or just the irrigated share regression (the one highlighted
in the letter).

\textcolor{blue}{We modified the pertinent sentences to now state the following:}

\textcolor{blue}{Note that irrigated area share and average crop yield estimates use data only from counties that have both irrigated and rainfed production observations because irrigated area share cannot be calculated when missing either one of the irrigated and rainfed observations. Counties with solely irrigated or rainfed production observation data (Section 4.1) are used solely for estimating impacts of drought on per-area crop irrigated or rainfed yields respectively (Equation 1).}

Please also provide a short rationale, e.g.~whether this is due to the
model selected or substantive considerations. Finally, please clarify
what this implies for the result interpretation. Fully irrigated county
observations do not seem obviously irrelevant. The change in Fig. 3
results due to the correction is quantitatively minor but not
irrelevant. It would, therefore, be helpful for readers to understand
whether the exclusion of fully irrigated counties makes the results more
accurate, or whether it creates a (necessary) bias.

\textcolor{blue}{Please see our replies above that provide detail on the rational behind this change. In summary, the change was made to avoid introducing assumptions about the irrigated area share for counties where rainfed production is not reported. It is not possible to calculate irrigated area share for counties with only irrigated or rainfed observations. We believe that ommitting counties lacking either of these data is preferable to prior assumptions made (e.g. that there is no irrigated or rainfed production if no data is reported in a county), as these assumptions may introduce noise into our analysis - for example if rainfed production is not reported due to sampling decisions made in the generation of estimates by USDA.}

Relatedly, it seems that in Fig. 7, some of the previously displayed
counties in both the irrigated (color shaded) and rainfed (grey)
categories are now missing. Did the exclusion of fully irrigated
counties also affect which rainfed areas are represented? If so, please
also clarify this link for the readers.

\textcolor{blue}{Thank you for noticing this change. This has nothing to do with the exclusion of counties with only irrigation acres reported. The maps were intended to present saturated thickness for "all" the counties used in the analysis. So, even the counties that were dropped for the irrigation share analsyis should have been present. The figure has been corrected to reflect this, and information added to the caption to explain this distinction.}

\begin{enumerate}
\def\labelenumi{\arabic{enumi}.}
\tightlist
\item
  \textcolor{blue}{In the 1st round, we made a correction to the lower bound of saturated thickness according to reviewer 1's comment. We stated that 9m was used as the lower bound as below.}
\end{enumerate}

\textcolor{blue}{"Aquifer thickness values were only estimated for counties with irrigated yield observations. Furthermore, during the aggregation process, any aquifer cells within a county that had a aquifer thickness value lower than 9 m were removed before aggregation to reflect the fact that aquifer thickness lower than this threshold are not considered to be viable for extraction and therefore it is unlikely these areas contribute groundwater for irrigation production [Fenichel et al., 2016, Haacker et al., 2016, Deines et al., 2020]."}

\textcolor{blue}{However, we were actually using 12m as the lower bound, which we realized after seeing reviewer's comment on this in the first round:}

\textcolor{blue}{"Figure 2 - Why do the plots stop at a thickness of about 12 m when you had earlier defined the minimum thickness as 9 m and mention 10 m in the text when discussing the relationships?"}

\textcolor{blue}{Therefore, we change the code so that 9m is used instead of 12m. We, however, did not update the map according to this change in the data. This is why the map in this version has several more counties in Kansas and New Mexico that did not appear in the map of the first round revision. They used to be dropped prior to chaging the threshold from 12m to 9m.}

\begin{enumerate}
\def\labelenumi{\arabic{enumi}.}
\setcounter{enumi}{1}
\tightlist
\item
  \textcolor{blue}{The significant change in the map in the second round from the 1st round is a mistake on using the map we created when we were experimenting with codes. We then did not rerun the code to create a map with the whole dataset to overwrite the wrong map with the correct map. In this round, we made sure that we streamlined all the codes, rerun all of them, and cofirmed that the map indeed reflect what we used for our analysis in this round. Thank you again for your keen eyes on details.}
\end{enumerate}

\end{document}
