% Options for packages loaded elsewhere
\PassOptionsToPackage{unicode}{hyperref}
\PassOptionsToPackage{hyphens}{url}
%
\documentclass[
]{article}
\usepackage{amsmath,amssymb}
\usepackage{lmodern}
\usepackage{url}
\usepackage{iftex}
\ifPDFTeX
  \usepackage[T1]{fontenc}
  \usepackage[utf8]{inputenc}
  \usepackage{textcomp} % provide euro and other symbols
\else % if luatex or xetex
  \usepackage{unicode-math}
  \defaultfontfeatures{Scale=MatchLowercase}
  \defaultfontfeatures[\rmfamily]{Ligatures=TeX,Scale=1}
\fi
% Use upquote if available, for straight quotes in verbatim environments
\IfFileExists{upquote.sty}{\usepackage{upquote}}{}
\IfFileExists{microtype.sty}{% use microtype if available
  \usepackage[]{microtype}
  \UseMicrotypeSet[protrusion]{basicmath} % disable protrusion for tt fonts
}{}
\makeatletter
\@ifundefined{KOMAClassName}{% if non-KOMA class
  \IfFileExists{parskip.sty}{%
    \usepackage{parskip}
  }{% else
    \setlength{\parindent}{0pt}
    \setlength{\parskip}{6pt plus 2pt minus 1pt}}
}{% if KOMA class
  \KOMAoptions{parskip=half}}
\makeatother
\usepackage{xcolor}
\usepackage[margin=1in]{geometry}
\usepackage{graphicx}
\makeatletter
\def\maxwidth{\ifdim\Gin@nat@width>\linewidth\linewidth\else\Gin@nat@width\fi}
\def\maxheight{\ifdim\Gin@nat@height>\textheight\textheight\else\Gin@nat@height\fi}
\makeatother
% Scale images if necessary, so that they will not overflow the page
% margins by default, and it is still possible to overwrite the defaults
% using explicit options in \includegraphics[width, height, ...]{}
\setkeys{Gin}{width=\maxwidth,height=\maxheight,keepaspectratio}
% Set default figure placement to htbp
\makeatletter
\def\fps@figure{htbp}
\makeatother
\setlength{\emergencystretch}{3em} % prevent overfull lines
\providecommand{\tightlist}{%
  \setlength{\itemsep}{0pt}\setlength{\parskip}{0pt}}
\setcounter{secnumdepth}{-\maxdimen} % remove section numbering
\ifLuaTeX
  \usepackage{selnolig}  % disable illegal ligatures
\fi
\IfFileExists{bookmark.sty}{\usepackage{bookmark}}{\usepackage{hyperref}}
\IfFileExists{xurl.sty}{\usepackage{xurl}}{} % add URL line breaks if available
\urlstyle{same} % disable monospaced font for URLs
\hypersetup{
  hidelinks,
  pdfcreator={LaTeX via pandoc}}

\author{}
\date{\vspace{-2.5em}}

\begin{document}

\section{Reply to reviewer \#1}

This manuscript assesses the impact of aquifer thickness on agricultural
production during drought in the area overlying the High Plains aquifer
in the central United States. The authors use crop, meteorological, and
aquifer thickness data in regressions to develop relationships among
these quantities. Their most important finding is the impact of aquifer
thickness on productivity during drought and the ramifications for
aquifer conservation efforts. I have read this manuscript over a number
of times and found it of considerable interest and worthy of publication
in Nature Water. However, I do have a few concerns that I wanted to
bring to the attention of the authors. My most important concerns are as
follows (not necessarily in order of importance):

\textcolor{blue}{Thank you for the detailed and constructive feedback on our manuscript. We have endeavored to address all your comments and suggestions, and our replies to your comments can found in blue below (original comments in black)}

\begin{enumerate}
\def\labelenumi{\arabic{enumi}.}
\tightlist
\item
  Aquifer thickness versus saturated thickness -- The authors speak of
  ``aquifer saturated thickness'', which is a redundant term. Remember, the definition of an aquifer is a saturated unit – there is no unsaturated aquifer. Please use ``aquifer
  thickness'', to avoid propagating this confusing usage into the
  future.
\end{enumerate}

\textcolor{blue}{We agree that there is some redundant use of words here. All the instances of ``saturated thickness'' and ``aquifer saturated thickness'' have been replaced by ``aquifer thickness'' in the revised manuscript.}

\begin{enumerate}
\def\labelenumi{\arabic{enumi}.}
\setcounter{enumi}{1}
\tightlist
\item
  Dependence on aquifer thickness -- I had a number of questions about
  the relationships displayed in Figure 1. First, the regression appears
  to have been done year by year. Are the quantiles recalculated each
  year to reflect that there may be changes between categories as time
  goes on? Second, when the authors state the lowest quantile goes from
  0 to 33\% are they assuming that the minimum thickness of 9 m is the
  zero point? Third, I assume that the county and year fixed effects are
  calculated as part of the regression. Are there any insights that can
  be gained from the interpretation of those quantities? Fourth, the
  authors need to point out that the same aquifer thickness can result
  in dramatically different transmissive characteristics. Thus, a
  relatively thin aquifer interval composed of coarse sands and gravels
  may be more resilient to drought than a much thicker aquifer dominated
  by fine sands and silts. Could that be a major player in the error
  bars in the plots? Please mention this at some point in the
  manuscript.
\end{enumerate}

\begin{itemize}
\item
  First point:
  \textcolor{blue}{Quantiles are not recalculated each year. If we do so, then the same quantile would represent different aquifer thickness range every year. When modeling the impact of aquifer thickness, this is not desirable because it would be expected that different ranges of aquifer thickness would have the same impact on a farmer's ability to buffer drought risk irrespective of year.}
\item
  Second point:
  \textcolor{blue}{Thank you for highlighting this point. For the 0\% quantile, we assume a aquifer thickness of 9m. This minimum of aquifer thickness was selected based on existing literature (Fenichel et al., 2016; Haacker et al., 2016; Deines et al., 2020), which typically defines 9m as the minimum aquifer thickness needed to provide a functioning water supply. We have modified the text on page 13 lines 298-302 of the revised manuscript to note these points.}
\item
  Third point:
  \textcolor{blue}{Yes, county and year fixed effects are calculated as part of the regressions. However, these variables can be interpreted simply as shifters that tell us how much higher yields in a single county are on average relative to another county and to account for trends in yields over time. As these variables do not interact with aquifer thickness in our analysis, there would be little insights gained from analyzing these variables - and indeed it is standard practice to not report and directly analyze fixed effects terms in econometric regressions used in statistical crop-climate modeling.}
\item
  Fourth point:
  \textcolor{blue}{This is an important point. You are correct that unobserved factors, such as aquifer heterogeneity influencing well yield responses to changes in aquifer thickness, are a likely underlying cause of the uncertainty ranges found in our analyses. Variations in aquifer conductivity, for example, would alter the relationship between saturated thickness and well capacity, thereby adding noise to the relationship between aquifer thickness and drought risk exposure. This noise is referred to as measurement error in econometric studies, and can result in attenuation bias. Specifically, measurement error in an explanatory variable (here, aquifer thickness) tends to bias your estimation in a way that makes the impact of the explanatory variable with measurement errors appear "less" influential than it truly is. So, in our context, our estimated impact of aquifer thickness on the share of irrigation is conservative because measurement error will be attenuating some of the true impact of aquifer thickness on drought risk exposure. Furthermore, our analysis shows that - even in the presence of these uncertainties - we are still able to identify statistically significant increases in drought risk exposure for irrigated agriculture as a function of declining aquifer thickness. This further reinforces the importance of aquifer conditions as a determinant of adaptive capacity to drought. To highlight these points, we have added the following text in the Discussion section on page 10 lines 213-227 of our revised manuscript:}
\end{itemize}

\textcolor{blue}{"Unobserved variations in aquifer properties will also introduce noise into the relationships between aquifer thickness, well yields, and drought risk. For example, a thinner aquifer interval composed of coarse sands and gravels may be able to supply comparable or higher well yields than a thicker aquifer dominated by fine sands and silt deposits, and hence offer greater resilience to water deficits despite similar or larger aquifer thickness [Butler et al., 2013, Korus and Hensen, 2020]. These measurement errors in the relationship between aquifer thickness and well yield will lead to attenuation bias [Bound and Krueger, 1991, Hyslop and Imbens, 2001], suggesting that our analysis is likely to be an underestimate of the true impacts of groundwater depletion on drought risk. Nonetheless, our analysis shows that - even in the presence of these measurement uncertainties - we are still able to identify statistically significant increases in drought risk exposure for irrigated agriculture as a function of declining aquiefr thickness. Improved data on spatial variations in aquifer properties and greater monitoring of real-world changes in well capacities alongside traditional water level measurements would help to further refine understanding of patterns of drought vulnerability across the HPA. This would enable more precise estimates to be made of critical tipping points in the resilience of irrigated agriculture to drought, which are needed to effectively design and target groundwater conservation policies."}

\textcolor{blue}{New references added:}

\begin{itemize}
\item
  \textcolor{blue}{Butler, J. J., Stotler, R. L., Whittemore, D. O., and Reboulet, E. C. (2013). Interpretation of water level changes in the High Plains aquifer in western Kansas. Groundwater, 51(2), 180-190.}
\item
  \textcolor{blue}{Korus, J. T., and Hensen, H. J. (2020). Depletion percentage and nonlinear transmissivity as design criteria for groundwater-level observation networks. Environmental Earth Sciences, 79(16), 382.}
\item
  \textcolor{blue}{John Bound and Alan B Krueger. The extent of measurement error in longitudinal earnings data: Do two wrongs make a right? Journal of labor economics, 9(1):1–24, 1991.}
\item
  \textcolor{blue}{Dean R Hyslop and Guido W Imbens. Bias from classical and other forms of measurement error. Journal of Business \& Economic Statistics, 19(4):475–481, 2001}
\end{itemize}

\begin{enumerate}
\def\labelenumi{\arabic{enumi}.}
\setcounter{enumi}{2}
\tightlist
\item
  Conditioned on drought risk exposure -- I wasn't sure what the authors
  meant by this in the second paragraph of Section 2.2. This must be
  related to the water deficit but the lack of explanation in the main
  text or the methods section left me wondering. Please clarify at some
  point in the manuscript.
\end{enumerate}

\textcolor{blue}{By "conditioned on drought risk exposure", we were referring to the fact that Figure 4 displays the relationship between aquifer thickness and the share of production area that is irrigated while also accounting for variations in climate across the aquifer. I.e. that the Figure shows the impact of aquifer thickness on share of production irrigated after removing the effects of climate on share of production irrigated. To improve clarity, we have modified this phrase to now state "...after accounting for spatio-temporal variations in weather conditions and drought risk exposure across the aquifer".}

\begin{enumerate}
\def\labelenumi{\arabic{enumi}.}
\setcounter{enumi}{3}
\tightlist
\item
  Equation 2 -- Please define all terms in the equation and provide an
  explanation of what is meant by ``estimated semi-parametrically'\,'.
\end{enumerate}

\textcolor{blue}{We now added a more detailed summary of the mathematical expressions of the model with all the terms explained in Section 4.2 of our revised manuscript. The word semi-parametrically refers to the fact that generalized additive models, such as used in our analysis, estimate the impact of a variable using splines without functional form assumption. These are sometimes referred to as semi-parametric regression methods. However, we agree that the term is potentially confusing, and therefore have removed the term "semi-parametrically" in our revised manuscript.}

\begin{enumerate}
\def\labelenumi{\arabic{enumi}.}
\setcounter{enumi}{4}
\tightlist
\item
  Kansas and California efforts -- I've seen a number of papers
  describing efforts in Kansas and California to address aquifer
  depletion via LEMAs (Kansas) and SGMA (California). The 1985-2016
  period examined by the authors was before most of the efforts in
  Kansas got started. My understanding is that the first LEMA was
  established in 2013 but then no more were set up until after 2016.
  This manuscript would be more impactful if the authors could speculate
  on how their relationships might change with greater attention to
  groundwater conservation. For example, I would think the decrease in
  yield the authors observe with negative water deficits would disappear
  when the irrigators are using soil moisture sensors and are focused on
  water-use efficiency. I would expect changes as well for tipping
  points.
\end{enumerate}

\textcolor{blue}{This is an interesting idea - thank you for raising it. We agree that efforts to improve aquifer management could help to mitigate some of the changes in productivity driven by climate and aquifer conditions in our analysis. Greater attention to groundwater conservation, and use of improved scheduling technologies such as soil moisture probes, would be expected to mitigate production losses in years with negative water deficits (i.e. wetter years) which we attribute to increased possibility for over-irrigation when well capacities are high (and hence when farmers do not face or observe physical scarcity). In years with large positive water deficits (i.e. drought years), improved irrigation water management would also be expected to mitigate some of the additional production risk caused by smaller aquifer thickness. However, we would still expect smaller aquifer thickness to result in elevated exposure to drought risk even in these circumstances, as declining well yields would create a physical limit to water supply which would stop farmers being able to fully meet crop water needs in these years. Formally testing these effects is beyond the scope and data of our study, but we agree that these are valuable areas for future work and have added the following text to the discussion of our revised manuscript on page 11 lines 233-246:}

\textcolor{blue}{"Improvements to irrigation scheduling practices and technologies, such as have been implemented as part of the SD6-LEMA (Local Enhanced Management Area) in Kansas (Deines et al., 2019; Glose et al., 2022) and in parts of Texas (Mrad et al., 2020) and which are proposed as part of California's Sustainable Groundwater Management Act (SGMA) (Berbel and Esteban, 2019; Lubell et al., 2020), could also help to improve the efficiency of irrigation and, hence, the ability to more effectively meet crop water demands with lower well capacities. Such interventions may reduce - but not eliminate - increases in drought risk caused by lower or declining aquifer thickness, and could also help to mitigate against potential yield losses caused by over-use of water in wetter years via raising awareness of efficiency saving potentials (Foster et al., 2019) and the importance of conservation (Marston et al., 2022). However, risks also exist that irrigation efficiency improvements will increase net extraction, ultimately causing a rebound effect that exacerbates future aquifer depletion and drought risks (Grafton et al., 2018; Perez-Blanco et al., 2021)."}

\textcolor{blue}{New references added:}

\begin{itemize}
\item
  \textcolor{blue}{Glose, T. J., Zipper, S., Hyndman, D. W., Kendall, A. D., Deines, J. M., and Butler, J. J. (2022). Quantifying the impact of lagged hydrological responses on the effectiveness of groundwater conservation. Water Resources Research, 58(7), e2022WR032295.}
\item
  \textcolor{blue}{Berbel, J., and Esteban, E. (2019). Droughts as a catalyst for water policy change. Analysis of Spain, Australia (MDB), and California. Global Environmental Change, 58, 101969.}
\item
  \textcolor{blue}{Lubell, M., Blomquist, W., and Beutler, L. (2020). Sustainable groundwater management in California: A grand experiment in environmental governance. Society and Natural Resources, 33(12), 1447-1467.}
\item
  \textcolor{blue}{Marston, L. T., Zipper, S., Smith, S. M., Allen, J. J., Butler, J. J., Gautam, S., and David, J. Y. (2022). The importance of fit in groundwater self-governance. Environmental Research Letters, 17(11), 111001.}
\item 
  \textcolor{blue}{Pérez-Blanco, C. D., Loch, A., Ward, F., Perry, C., and Adamson, D. (2021). Agricultural water saving through technologies: a zombie idea. Environmental Research Letters, 16(11), 114032.}
\item 
  \textcolor{blue}{Grafton, R.Q., Williams, J., Perry, C.J., Molle, F., Ringler, C., Steduto, P., Udall, B., Wheeler, S.A., Wang, Y., Garrick, D. and Allen, R.G. (2018). The paradox of irrigation efficiency. Science, 361(6404), 748-750.}
\item 
  \textcolor{blue}{Mrad, A., Katul, G.G., Levia, D.F., Guswa, A.J., Boyer, E.W., Bruen, M., Carlyle-Moses, D.E., Coyte, R., Creed, I.F., Van De Giesen, N. and Grasso, D. (2020). Peak grain forecasts for the US High Plains amid withering waters. Proceedings of the National Academy of Sciences, 117(42), 26145-26150.}
\end{itemize}

\hypertarget{minor-points}{%
\subsection{Minor points}\label{minor-points}}

\begin{itemize}
\tightlist
\item
  Section 1, 3rd Paragraph -- ``Two aquifer pathways exist through aquifer depletion will.'' – confusing sentence – please replace ``through'' with ``by which''
\end{itemize}

\textcolor{blue}{We have corrected the text in line with the reviewer's comment.}

\begin{itemize}
\tightlist
\item
  Section 1, 4th Paragraph --``we generate empirical evidence'' seems awkward to me. How about replacing with ``we present empirical evidence''? Next sentence – ``Our analysis advances on previous $\dots$'' reads better as ``Our analysis extends previous $\dots$'' There are many issues such as these and the previous comment. Please address such issues so that the manuscript is an easier read.
\end{itemize}

\textcolor{blue}{We completely agree. Both of the suggested changes have been made.}

\begin{itemize}
\tightlist
\item
  Section 2.1 title -- I would delete ``marginally'\,' as it is not
  needed and some will find it confusing.
\end{itemize}

\textcolor{blue}{We agree and have removed this term for clarity.}

\begin{itemize}
\tightlist
\item
  Figure 2 - Why do the plots stop at a thickness of about 12 m when you
  had earlier defined the minimum thickness as 9 m and mention 10 m in
  the text when discussing the relationships?
\end{itemize}

\textcolor{blue}{Thank you for catching this typo. The correct value is 9 m and the text has been modified to reflect this and ensure consistency. Plots have also been updated to finish at an aquifer thickness of 9 m where relevant.}

\begin{itemize}
\tightlist
\item
  Section 2.3, First Paragraph -- please clarify that ``average per-area
  production'\,' is a combination of irrigated and rainfed yields. That
  would make the manuscript easier to follow.
\end{itemize}

\textcolor{blue}{We have modified the text in Section 2.3, first paragraph to now more clearly state: "weighted-average of irrigated and rainfed yields, where weights are production area shares".}

\begin{itemize}
\tightlist
\item
  Figure 5 -- I assume that the left plot is corn and the right soybean,
  but please clarify.
\end{itemize}

\textcolor{blue}{Thanks for noticing this. We have added this detail to the figure and its caption.}

\begin{itemize}
\tightlist
\item
  Data and Code Availability -- Are the data from Dr.~Haacker on the
  Github repository? Please place them there prior to publication so
  future investigators can access them rather than having to track down
  Dr.~Haacker. The data set is over seven years old so it should be made
  available according to the principles of open science.
\end{itemize}

\textcolor{blue}{The aquifer thickness data from Dr. Haacker are in the public and can be accessed from the following link: \url{https://www.hydroshare.org/resource/7d925c7944244032af98c9ed20c22db6/}. We have added this information to our data and code availability section of our revised manuscript.}

\begin{itemize}
\tightlist
\item
  Line before Equation (2) -- ``REF'\,' should be replaced with the
  actual reference.
\end{itemize}

\textcolor{blue}{The missing references have been added as follows: "[Foster et al., 2015a,b, Hrozencik et al., 2017]".}

\begin{itemize}
\tightlist
\item
  Discussion, First Paragraph -- ``reduction in aquifer depletion'' should be ``reduction in aquifer thickness''.
\end{itemize}

\textcolor{blue}{Thank you for catching this. This typo has been corrected in the revised manuscript.}

\begin{itemize}
\tightlist
\item
  Section 4.2, 3rd Paragraph -- ``the the impact'\,' -- remember to
  spell and grammar check prior to submission.
\end{itemize}

\textcolor{blue}{Thank you for highlighting this error. We have amended the text and also done a further proofread of our manuscript to capture and address and further spelling and grammatical errors.}

\begin{itemize}
\tightlist
\item
  Reference in Section 3 -- I doubt that ``Butler Jr'' is the surname, please replace with ``Butler''. Suffixes apparently aren't handled well by some of the bibliography generation programs.
\end{itemize}

\textcolor{blue}{The reference has been updated in the revised manuscript}

\begin{itemize}
\tightlist
\item
  Please remember to close all parentheses and don't mispair brackets
  and parentheses. Such stuff can raise questions about the care used in
  the analysis.
\end{itemize}

\textcolor{blue}{All parentheses and brackets have been checked and corrected where necessary in the revised manuscript}

\section{Reply to reviewer \#2}

\textcolor{blue}{Thank you for the detailed and constructive feedback on our manuscript. We have endeavored to address all your comments and suggestions, and our replies to your comments can found in blue below (original comments in black).}

This is an interesting paper that asks an important yet answered
question in the field of agricultural sustainability - how much is
groundwater depletion reducing agricultural productivity in the High
Plains region in the United States? While the paper is an important
contribution to the literature, there are several issues that need to be
addressed prior to acceptance at any journal.

\begin{enumerate}
\def\labelenumi{\arabic{enumi}.}
\tightlist
\item
  There needs to be more description about why aquifer saturation
  thickness is an appropriate proxy for farmers' ability to irrigate
  from wells. This is particularly true since this is a broad interest
  journal, and the mechanism for the main independent variable should be
  very clear to all readers.
\end{enumerate}

\textcolor{blue}{Aquifer thickness affects farmers ability to meet the water requirements of their crops due to its impacts on the pumping yields (or capacities) of groundwater irrigation wells. As the thickness of an aquifer declines, the rate at which water can flow from the aquifer to the pumping well is reduced proportionally to the aquifer's transmissivity (equal to the hydraulic conductivity of the aquifer geology multiplied by the saturated thickness of that aquifer geology). For small relative changes in aquifer thickness (e.g. for initial depletion of an aquifer with large initial thickness), the change in well yield is small and unlikely to have substantial negative impacts on drought risk exposure as a farmer could simply run their pump for slightly longer to compensate for the small reduction in instantaneous well yield. However, for larger relative changes in aquifer thickness (e.g. due to significant aquifer depletion and/or thinner initial aquifer bodies), the magnitude of well yield reductions will be much larger. These reductions in water supply capacity cannot be mitigated by changes to irrigation scheduling or management practices. The farmer will often be unable to keep up with crop water demands during the peak of the growing season in the absence of rainfall, even if running their system continually. This effect can also typically not be mitigated by drilling additional or deeper wells, due to a combination of both regulatory moratoriums on well drilling - which are common across the HPA - and high costs of deep well construction. To more clearly highlight for non-specialist readers these connections between saturated thickness, well yields, and drought resilience of agriculture, we have added the following text on page 1 lines 40-47 of our revised manuscript. We note that we are limited in elaborating further on background information given journal word constraints, and hence provide key supporting studies in citations for readers interested in deepening understanding of these dynamics.}

\textcolor{blue}{"...depletion also reduces the thickness of an aquifer, lowering the aquifer's transmissivity (i.e., the potential rate of water supply to wells). This latter effect reduces the rate at which water can be pumped and applied to a field as irrigation (Konikow and Kendy, 2005; Foster et al., 2015; Hrozencik et al., 2017). If the reduction in well yields is substantial - for example as would occur when a large proportion of aquifer thickness is depleted (Hecox et al., 2002; Korus and Hensen, 2020) - then a farmer may be unable to extract sufficient water to meet crop water demands, in particular during periods of limited rainfall and in the peak of the growing season when crops are most sensitive to water deficits (Foster et al., 2015a; Rouhi-Rad et al., 2020)."
}

\textcolor{blue}{Additional references added:}

\begin{itemize}
\item
  \textcolor{blue}{Hecox, G.R., Macfarlane, P.A., and Wilson B.B. (2002). Calculation of yield for High Plains wells: relationship between saturated thickness and well yield. Open file report 2002–25C, Kansas Geological Survey: Lawrence, Kansas, USA.}
\item
  \textcolor{blue}{Korus, J. T., and Hensen, H. J. (2020). Depletion percentage and nonlinear transmissivity as design criteria for groundwater-level observation networks. Environmental Earth Sciences, 79(16), 382.}
\item
  \textcolor{blue}{Rouhi-Rad, M., Araya, A., and Zambreski, Z. T. (2020). Downside risk of aquifer depletion. Irrigation Science, 38, 577-591.}
\end{itemize}

\begin{enumerate}
\def\labelenumi{\arabic{enumi}.}
\setcounter{enumi}{1}
\tightlist
\item
  Though I am not trained as a hydrologist, my understanding is that
  aquifer saturation thickness is a good measure of general water
  availability in an aquifer but not necessarily how much water is
  available at a specific location or point. If this is the case, the
  authors should make clear why this is an appropriate measure to use in
  their study and potential limitations of using this approach.
\end{enumerate}

\textcolor{blue}{Aquifer thickness is a spatially and temporally variable property. It  represents the variation in water storage at specific places in an aquifer and at specific points in time, not the overall aggregated storage within an aquifer. Aquifer thickness also influences the transmissivity of an aquifer, which in turn is proportional to well yields. Aquifer thickness therefore influences the potential rate of water extraction at different points within an aquifer and how this varies over time (e.g., due to depletion reducing aquifer thickness). As noted in our reply to another reviewer, it is important to also note that the relationship between aquifer thickness and well yields will also be influenced by spatial variations in aquifer geology (specifically hydraulic conductivity which together with saturated thickness defines aquifer transmissivity). Data on county-level spatial variations in aquifer hydraulic conductivity is not available for HPA (or indeed for other aquifers in general), meaning we cannot control for these variations thus adding noise or uncertainy to our analyses. This noise is referred to as measurement error in econometric studies, and can result in attenuation bias. Specifically, measurement error in an explanatory variable (here, aquifer thickness) tends to bias your estimation in a way that makes the impact of the explanatory variable with measurement errors appear "less" influential than it truly is. So, in our context, our estimated impact of aquifer thickness on the share of irrigation is conservative because measurement error in the relationship between aquifer thickness and well yields will be attenuating some of the true impact of aquifer thickness on drought risk. To highlight these points, we have added the following texts in the Discussion section on page 10 lines 210-224 of our revised manuscript.}

\textcolor{blue}{"Unobserved variations in aquifer properties will also introduce noise into the relationships between aquifer thickness, well yields, and drought risk. For example, a thinner aquifer interval composed of coarse sands and gravels may be able to supply comparable or higher well yields than a thicker aquifer dominated by fine sands and silt deposits, and hence offer greater resilience to water deficits despite similar or larger aquifer thickness [Butler et al., 2013, Korus and Hensen, 2020]. These measurement errors in the relationship between aquifer thickness and well yield will lead to attenuation bias [Bound and Krueger, 1991, Hyslop and Imbens, 2001], suggesting that our analysis is likely to be an underestimate of the true impacts of groundwater depletion on drought risk. Nonetheless, our analysis shows that - even in the presence of these measurement uncertainties - we are still able to identify statistically significant increases in drought risk exposure for irrigated agriculture as a function of declining saturated thickness. Improved data on spatial variations in aquifer properties and greater monitoring of real-world changes in well capacities alongside traditional water level measurements would help to further refine understanding of patterns of drought vulnerability across the HPA. This would enable more precise estimates to be made of critical tipping points in the resilience of irrigated agriculture to drought, which are needed to effectively design and target groundwater conservation policies."}

\begin{enumerate}
\def\labelenumi{\arabic{enumi}.}
\setcounter{enumi}{2}
\tightlist
\item
  It is unclear to me why county and time fixed effects were not used in
  every regression. In particular, it does not look like they were
  included in the regression that modeled irrigated area as a function
  of saturated thickness and water deficit. Yet, accounting for
  time-invariant differences across counties is important, especially
  because this is the regression that showed the largest effect of
  saturated thickness on a measure of agricultural production (irrigated
  area). Is there not enough variation within a county through time to
  pick up the effect of saturated thickness on irrigated area? This
  should all be made clearer, and a model that better captures the
  causal relationship between saturated thickness and area irrigated
  should be used if possible. If it's not possible, the limitations of
  the current approach should be made very clear and potential issues
  with inference.
\end{enumerate}

\textcolor{blue}{Thank you for this comment, which we believe is an important point to clarify in terms of our methods and the assumptions made in our analysis. County and year fixed effects were included in equation (1), but only year fixed effects were included in equation (2). As you guessed, the problem of including county FEs lies in the fact that this would eliminate a large proportion of the variation in aquifer thickness, which is predominantly cross-sectional rather than longitudinal. To illustrate this, Figure 6 in Appendix A provides histograms of (1) aquifer thickness demeaned by the county average, (2) aquifer thickness demeaned by the state average, and (3) aquifer thickness without demeaning. As shown in Figure 6, inclusion of county FEs (mathematically equivalent to regression using county-demeaned aquifer thickness) dramatically reduces the total variation in aquifer thickness leaving little variations to accurately estimate the impact of aquifer thickness on the share of irrigated acres. The inclusion of State FEs does not diminish the variation in aquifer thickness. Therefore, while the inclusion of State FEs does not provide as tight of a control as county FEs, we have decided to include State FEs for both corn and soybean irrigated share regressions to provide a balance between preserving spatial variations in aquifer thickness and controlling for other spatially dependent effects. Figure 7 compares the estimated impacts of aquifer thickness with or without State FEs. The negative impact of reduction in aquifer thickness on the share of irrigated acres is slightly greater when state fixed effects are included. This suggests that aquifer thickness is unlikely to be very highly correlated with important unobservable variables that are affecting irrigated share significantly. Given the direction of potential bias, providing tighter controls through inclusion of county fixed effects therefore may result in even more significant drop in the share of irrigated acres with smaller aquifer thickness suggesting our results are possibly a conservative estimate of the true impact of aquifer thickness on drought risk. A detailed discussion of these analyses and tradeoffs is now provided in Appendix A.
}

\begin{enumerate}
\def\labelenumi{\arabic{enumi}.}
\setcounter{enumi}{3}
\tightlist
\item
  There are minor typos (e.g., missing reference listed as REF) that
  should be corrected prior to resubmission.
\end{enumerate}

\textcolor{blue}{The missing in-text citation has been corrected, and the revised manuscript has been thoroughly proofread to catch any remaining spelling, grammatical, and formatting errors.}


\section{Reply to reviewer \#3}


The paper aims to show the impacts of aquifer depletion on irrigated
crop yields and production areas in the United States High Plains. To do
so, the paper applies a panel data analysis approach to a detailed
county-level dataset of saturated thickness, water deficit, yields, and
land-use, using a spline model. The authors find that aquifer depletion
reduces farmers' ability to cope with water deficits.

The significance and novelty of the paper stems from the clear and
encompassing empirical approach to the subject matter. The depletion of
the Ogallala aquifer and implications for irrigated agriculture are not
new per se. A compelling quality of the paper is the clear statistical
evaluation of a large panel dataset, constructed to show the negative
impacts of aquifer depletion on farmer's ability to cope with water
deficits, over a long period of time. To fully harness the potential of
this quality for readers, the authors should mainly consider improving
the clarity of the figures presented and of their description in the
text, and addressing some other points that remained unclear (see
details below).

\textcolor{blue}{Thank you for the detailed and constructive feedback on our manuscript. We have endeavored to address all your comments and suggestions, and our replies to your comments can found in blue below (original comments in black).}

\hypertarget{main-points}{%
\section{MAIN POINTS:}\label{main-points}}

\begin{enumerate}
\def\labelenumi{\arabic{enumi}.}
\tightlist
\item
  Abstract:
\end{enumerate}

\begin{itemize}
\tightlist
\item
  ``We show that aquifer depletion reduces the ability of farmers to
  sustain irrigated crop yields and production areas in drought years.''
  - If stated this way, please clarify in the results and/or discussion,
  how causality (direction of causality, no omitted factors) is
  established.
\end{itemize}

\textcolor{blue}{Thank you for raising this important point - we agree that clarifying the extent to which we can infer causality from our analysis is important. We discuss the underlying direction of causality in detail in the main section of our analysis, highlighting the biophysical mechanisms through saturated thickness of an aquifer influences well yields and how well yields will in turn affect drought risk exposure for a farmer. This discussion has been expanded on page 1 lines 37-54 to provide more detail on the physical mechanisms, and supporting empirical and theoretical evidence, connecting saturated thickness, well yields, and water-related crop production risks. For these reasons, we believe that our results and interpretation of our findings are strongly supported by and consistent with hydrologic and economic theories, and provide useful insights for groundwater management even if causality cannot be formally determined.}

\textcolor{blue}{We acknowledge that a further important factor when seeking to establish causality is the issue of possible bias in the estimation of the impact of a variable interest - in our case the effects of saturated thickness on agricultural drought risk. We do not intend to claim that our estimation is free of bias. Indeed, in all empirical studies, it is almost impossible to have no unobserved (omitted) factors unless the data is from a randomized control trial. Other sources of bias (e.g., model misspecification error) will also almost always exist. In our analysis, one of the main limitations is not including county fixed effects in our regressions, which would have controlled all the county-specific time-invariant characteristics. County-level fixed effects were not included because the majority of the variation in saturated thickness is cross-sectional rather than time-dependent, and incorporating county fixed effects would therefore have removed this variation. We acknowledge this as a limitation, and have added a detailed discussion in both the Methods (page 15 lines 339-344) and Appendix C of our revised manuscript.}

\textcolor{blue}{Furthermore, our analysis will also be influenced by measurement errors in aquifer thickness and its relationship with  well yield (e.g., due to spatially variable hydraulic conductivity of aquifer layers). This limitation is highlighted (with expanded text compared to our original manuscript) in the Discussion section (page 10 lines 210-220). Importantly, these measurement errors would be expected to result in attenuation bias, making our estimates of the impact of aquifer thickness smaller in magnitude than the likely true effect.}

\begin{itemize}
\tightlist
\item
  ``\ldots{} in drought years'' - Please define in the in the main text
  (or methods) how drought and drought years are defined. E.g. are any
  effects linked to increasing water deficits considered drought-related
  or only those above a certain level?
\end{itemize}

\textcolor{blue}{In this paper, we do not seek to formally define a threshold level of water deficit beyond which a year is considered to be a `drought'. Instead, we define meteorological conditions during the growing season in terms of a common water deficit metric - which provides a continuous measure of how `drought' a given year was in a specific county considering both precipitation and crop evapotranspiration demands. A water deficit of 1000 mm, for example, could be considered to be a "severe" drought as precipitation is substantially less than rainfall with the result that corn and soybean yields are estimated to be almost zero under rainfed production. In contrast, a water deficit of -200 mm would be considered not to be a drought because precipitation exceeds crop evapotranspiration, and as a result irrigated and rainfed yields are almost identical. We do agree that the wording "in drought years" may be somewhat confusing in this context given the lack of a formal threshold beyond which a year is considered to be drought. Therefore, we have amended this phrase throughout the manuscript to instead say "in years with large water deficits".}

\begin{enumerate}
\def\labelenumi{\arabic{enumi}.}
\tightlist
\item
  Figure 1: This figure does not convey the effects of saturated
  thickness very clearly, either because the underlying analysis does
  not produce substantial effects, or because they are not presented in
  a compelling manner. It might also be helpful to use a visualization
  that more effectively illustrates the statements in the text. See
  detailed points below:
\end{enumerate}

\textcolor{blue}{We appreciate this comment. In response, we have added an additional figure - Figure 2 in the revised manuscript - which shows in more detail the difference in yield between the 2nd (medium level of saturated thickness) and 3rd (highest level of saturated thickness) quantiles relative to the 1st quantile (lowest level of saturated thickness). This illustrates the differences in irrigated corn and soybean yields across different aquifer thickness levels. Please see our detailed responses to your comments below for more explanation of the interpretation of this figure with regard to distinguishing significant and non-significant results from our analysis.}

\begin{itemize}
\tightlist
\item
  Please distinguish more clearly between significant and
  non-significant results. P. 2 states that ``counties with the highest
  level of saturated thickness \ldots{} experience \ldots{} productivity
  close to rainfed yields''. From the figure it is difficult to tell if
  the 2nd and 3rd quantile are significantly different for soybean. For
  corn this does not seem to be the case. Similarly, the text at the
  bottom of p.~2 states that ``saturated thickness has a statistically
  significant impact on per-area irrigated crop yields in moderate to
  extreme drought conditions''. Please clarify how the significance is
  evaluated here. Which range of deficits is meant by ``moderate to
  extreme''? Are the soybean yields considered significantly different
  between quantiles? Are only the 1st and 3rd quantile considered
  significantly different for corn? - It could even be of interest to
  readers which results are not significant here, in contrast to the
  greater significance of the effects analyzed in Fig. 2 and Fig. 3.
\end{itemize}

\textcolor{blue}{We have broken the above comment into separate points, which we answer  individually below.}

Your comment: Please distinguish more clearly between significant and
non-significant results.

\textcolor{blue}{
Figure 2 in our revised manuscript shows the difference in irrigated crop yields between the 2nd (medium level of saturated thickness) and 3rd (highest level of saturated thickness) quantiles relative to the 1st quantile (lowest level of saturated thickness). The shaded area denotes the 95$\%$ confidence interval around these response curves - indicating whether the difference in irrigated yield between the different levels of saturated thickness is significant at at least the 5$\%$ level. From Figure 2, it is clearly seen that large water deficits - equivalent to a severe drought event as noted above - lead to irrigated corn and soybean yields that are significantly lower for both corn and soybean when saturated thickness is low compared to when saturated thickness is high. The size of the effect is greatest for corn compared to soybean, and when comparing the third quantile (highest saturated thickness) to the first quantile (lowest saturated thickness). As discussed in our original manuscript and expanded upon in the revised manuscript (page 2 lines 86-108), this reflects the larger water requirements and drought sensitivity of corn along with the non-linearities in the response of well yields to saturated thickness.
}

Your comment: P. 2 states that ``counties with the highest level of
saturated thickness \ldots{} experience \ldots{} productivity close to
rainfed yields''. From the figure it is difficult to tell if the 2nd and
3rd quantile are significantly different for soybean. For corn this does
not seem to be the case.

\textcolor{blue}{
You are right that there are only small differences between yields for both corn and soybean when comparing the second and third quantiles of saturated thickness. As noted in the manuscript, this reflects the non-linearities in the relationships between saturated thickness and well yield, which mean that impacts of aquifer depletion on drought risk are not typically felt until saturated thickness has been lowered sufficiently to reduce well yields to levels that impair ability to meet crop water demands. However, in this part of the text, our intention was not to distinguish the 2nd and 3rd quantiles. Instead, we were seeking to point out that - in years with low levels of water deficit (i.e., wetter or non-drought years) - counties with higher aquifer thickness (2nd and 3rd) have lower yields than counties that have the lowest aquifer thickness (1st quantile). Furthermore, yields of counties with higher levels of saturated thickness are comparable to that of rainfed production - in particular for soybean production. To more clearly highlight these points, we have modified the phrasing of the text (page 2 lines 95-108 in the revised manuscript) to now state:}

\textcolor{blue}{"At low water deficits (below 200mm), Figure 2 illustrates that counties with the highest levels of aquifer thickness (i.e., 3rd quantile in Figure 1) experience statistically lower yields for corn and soybean than counties with the lowest level of aquifer thickness (1st quantile). In case of soybean, irrigated yields for the lowest aquifer thickness category converge to levels comparable to those obtained from rainfed production."}

Your comment: Similarly, the text at the bottom of p.~2 states that
``saturated thickness has a statistically significant impact on per-area
irrigated crop yields in moderate to extreme drought conditions''.
Please clarify how the significance is evaluated here. Which range of
deficits is meant by ``moderate to extreme''?

\textcolor{blue}{We agree that our manuscript would benefit from greater clarity around wording on significance of aquifer depletion impacts on productivity at different levels of drought. In our revised manuscript, we have therefore replace words like "moderate to extreme" with more specific water deficit thresholds/ranges. For the phrase quoted by the reviewer, the text in the revised manuscript (page 2 lines 87-88) has been modified to now state:``in years with water deficit higher than about 750 mm.'' We have also explicitly added the term ``statistically'' when we are talking about statistical significance, as opposed to economic or agronomic significance (i.e. implications) of the results.
}

Your comment: Are the soybean yields considered significantly different
between quantiles? Are only the 1st and 3rd quantile considered
significantly different for corn? - It could even be of interest to
readers which results are not significant here, in contrast to the
greater significance of the effects analyzed in Fig. 2 and Fig. 3.

\textcolor{blue}{As discussed in our responses to earlier comments, Figure 2 and the revised supporting text added to our revised manuscript (page 2 lines 86-94) now provide an in-depth analysis of the statistical significance of yield impacts between different saturated thickness categories. For corn, Figure 2 shows that there are statistically significant differences in yields between the 3rd and 1st quantiles of saturated thickness, and also between the 2nd and 1st quantiles of saturated thickness in the case of extreme water deficits. For soybeans, yields are not significantly different across the three saturated thickness categories - likely due to soybeans having lower water requirements and therefore lower vulnerability to reduced well yield linked to smaller saturated thickness. We have added the following short statement about the insignificance of the impact of saturated thickness for per-area soybean yields on page 2 lines 91-94 of the revised manuscript:}

\textcolor{blue}{"Unlike corn, irrigated soybean yields are not statistically distinguishable among the three aquifer thickness groups. This is likely because soybean is not as water demanding a crop as corn, meaning that even counties with lower saturated thickness can meet the majority of crop water demands as well yield is not a significantly binding constraint on irrigation management decisions."}

\begin{itemize}
\tightlist
\item
  In light of point (1.a), please also consider whether additional
  panels showing tests for specific hypotheses or other visual aids
  could make the Fig. 1 more accessible to readers with regards to the
  statements made. The figure is quite crowded, with many confidence
  bands overlapping. Currently, it takes time for readers to parse where
  the confidence bands for some of the plot lines end. The figure could
  be of greater benefit to readers if it was more accessible.
\end{itemize}

\textcolor{blue}{Thank you for this valuable suggestion. As we have discussed above, we have added a new Figure 2, which provides a more detailed comparison of yield responses for the higher levels (2nd and 3rd quantiles) of saturated thickness compared to the lowest levels (1st quantile) of saturated thickness. We believe this change has made it significantly easier for the readers to understand the impact of saturated thickness on farmers ability to buffer crops against water deficits and drought. For Figure 1, it is important to have all the four curves in the same panel as that would allow the readers to compare across both rainfed and the three irrigated yield categories. Confidence intervals around the curves were removed for visual interpretation, taking on board your comments made about difficulty interpreting the original Figure 1. Instead, we have displayed the estimated curves with confidence intervals individually in separate panels in Figure A1 for corn and A2 for soybean of Appendix A.}

\begin{itemize}
\tightlist
\item
  How were confidence bands determined? Were additional tests performed
  to evaluate their validity? Please consider showing individual data
  points.
\end{itemize}

\textcolor{blue}{Confidence bands are derived from the standard error of the coefficients in the model represented by Equation 1. This is a standard approach for determining confidence intervals from a statistically estimated model, and evaluating alternate measures of defining confidence intervals is beyond the scope of our study. Instead, the role of these confidence intervals is to quantify the uncertainty of the point estimates using standard statistical approaches.}

\begin{itemize}
\tightlist
\item
  P. 2 states that ``for water deficits above around 400mm, yield
  changes become negative for counties with the lowest levels of
  saturated thickness (i.e., 1st quantile in Figure 1). This reflects
  the restrictions that lower saturated thickness and associated
  reductions in well yields place on farmers ability to fully meet crop
  water requirements during periods of more severe or extended
  precipitation deficits''. How valid is this interpretation in view of
  the fact that these counties see a similarly large yield decrease
  below 400mm and that counties with the highest levels of saturated
  thickness even see a much stronger increase in yields with increasing
  deficits?
\end{itemize}

\textcolor{blue}{We believe our interpretation is valid, but accept that this is a hypothesis and could require further testing to substantiate the effect that is beyond the scope of our data. We attribute yield reductions with declining saturated thickness at low levels of water deficit (i.e. wetter conditions) to possible over-irrigation by farmer. Over-irrigation in wetter years has been documented empirically by previous research in the High Plains Aquifer (e.g., Foster et al., 2019; Gibson et al., 2017; Gibson et al., 2019), and excess water is known to negatively affect grain yields through a variety of mechanisms including physical damage to crops, nutrient leaching, delays to planting and harvesting, and other factors (e.g., Li et al., 2019). Reductions in yields in this scenario could also reflect the lower availability of positive heat in wetter years, which would diminish the positive benefits of irrigation and possibly explain some of the convergence to rainfed yields. This alone would not explain the fact that yield declines for low water deficits are greatest for counties with the largest aquifer thickness (and by implication greatest potential to over-irrigate). To reflect these different possible interpretations and their uncertainties, we have modified the text on page 2 lines 95-108 of our revised manuscript to now state:}

\textcolor{blue}{"At low water deficits (below 200mm), Figure 2 illustrates that counties with the highest levels of aquifer thickness (i.e., 3rd quantile in Figure 1) experience statistically lower yields for corn and soybean than counties with the lowest level of aquifer thickness (1st quantile). In case of soybean, irrigated yields for the lowest aquifer thickness category converge to levels comparable to those obtained from rainfed production. These counter-intuitive results reflect a combination of two factors. First, in years with small water deficits there is less need for irrigation and hence irrigated and rainfed yield differences are reduced. Second, years with small or even negative water deficits are associated with wetter conditions during the growing season, which can negatively impact yield if waterlogging occurs leading to greater risks of physical crop damage (e.g., lodging of grains), delays to planting/harvesting, nutrient leaching, and restricted root growth in waterlogged soils (Li et al., 2019). Declines in irrigated yields could be indicative of over-irrigation by farmers, which has been shown to occur in the HPA region in wet years (Foster et al., 2019; Gibson et al., 2017; Gibson et al., 2019), exacerbating these risks. However, the small declines in irrigated yields at low or negative levels of water deficit may also reflect the fact that wetter years are typically cooler, reducing positive heat effects on crop growth and yield formation."}

\textcolor{blue}{New reference added:}

\textcolor{blue}{- Li, Y., Guan, K., Schnitkey, G. D., DeLucia, E., and Peng, B. (2019). Excessive rainfall leads to maize yield loss of a comparable magnitude to extreme drought in the United States. Global Change Biology, 25(7), 2325-2337.}

\begin{itemize}
\tightlist
\item
  Please clarify in the legend or caption that the quantiles indicate
  levels of saturated thickness.
\end{itemize}

\textcolor{blue}{Thank you for this comment. We made changes to all the relevant figures and their captions according to this suggestion.}

\begin{enumerate}
\def\labelenumi{\arabic{enumi}.}
\setcounter{enumi}{2}
\tightlist
\item
  Figure 2: Please clarify in the text how the significance of
  reductions in the share of irrigated acres is evaluated (e.g.~relative
  to which aquifer depletion level(s)?), and at which level of aquifer
  depletion the reduced irrigation share becomes significant for each
  crop.
\end{enumerate}

\textcolor{blue}{We have added a new figure in Appendix B - Figure B.1 - that presents the estimated difference in the share of irrigated acres at different values of aquifer thickness levels relative to an aquifer thickness of 130 m (equivalent to a very thick aquifer in our third quantile group) including 95\% confidence intervals. This figure shows that the reduction in irrigated acres is statistically significant at the 5\% level for aquifer thickness of below approximately 35 m for corn and 30 m for soybean. Given space constraints imposed for the main body of our manuscript, we decided to place this figure in the appendix as it is supplementary to the main discussion.}

\begin{enumerate}
\def\labelenumi{\arabic{enumi}.}
\setcounter{enumi}{3}
\tightlist
\item
  Figure 3:
\end{enumerate}

\begin{itemize}
\tightlist
\item
  Please clarify why this figure has no confidence bands. The text
  describes it as a combination of the effects shown in Fig. 1 and Fig.
  2, which both have confidence bands. Please clarify how significance
  is evaluated.
\end{itemize}

\textcolor{blue}{
Thank you for highlighting this. To maintain consistency with our response to your earlier comments and valuable points about ensuring clarity of our figures, we have decided to not present confidence intervals for this figure as in the case Figure 1 to avoid overcluttering the plot. Instead, we now have added a new Figure 5, which is similar to Figure 2. It compares average (irrigated and rainfed combined) yield for counties with aquifer thickness of 50 m, 90 m, and 130 m against those with 10m. As in Figure 2, the shaded area denotes the 95\% confidence interval and can be used to interpret significance at the 5\% level. Since these numbers are equidistant, they allow us to illustrate non-linearity in the impact of aquifer thickness. Figure 5 shows that for for 50 m, 90 m, and 130 m aquifer thickness levels, yields are significantly higher than for counties with 10 m of aquifer thickness for water deficits above 400 mm for corn and above around 600 mm for soybean. The magnitude of yield gains is greatest for 90 m and 130 m aquifer thickness, but somewhat lower for aquifer thickness of 50 m. These responses indicate the non-linearities in the response of yields to both water deficits and aquifer conditions, which we have also expanded discussion of in the text in the final paragraph of Section 2.3 on page 7 lines 154-164 of our revised manuscript
}

\begin{itemize}
\tightlist
\item
  Relatedly, the text states that deficits beyond 200mm yield
  significant reductions in corn production. 200mm is where the curves
  intersect. Please clarify if this actually is also where significance
  starts, or if significance starts at a higher deficit.
\end{itemize}

\textcolor{blue}{You are correct that the text in the original manuscript referred to the point at which the curves intersect. The threshold level of water deficit at which saturated thickness has statistically significant impact on the response of per-area corn production is approximately 350 mm. We believe this is now more clearly demonstrated in Figure 5, and have also clarifies this point in the revised text linked to this figure (paragraph 2 of section 2.3 on page 6).}

\begin{itemize}
\tightlist
\item
  Please double-check if the y-axis label is correct. The label (``yield
  (tonne/ha)'') is the same as for Fig. 1, which makes it difficult to
  distinguish the different results presented in Fig. 1 and 3.
\end{itemize}

\textcolor{blue}{Yield (tonne/ha) is correct. Figures 4 (previously 3) and 5 (new figure) both report yield. Figures 1 and 2 report "irrigated" and "rainfed" yields separately, while Figures 4 and 5 report "average" yields reflecting the combination of irrigated and rainfed yields conditional on estimated irrigated production shares.}

\begin{enumerate}
\def\labelenumi{\arabic{enumi}.}
\setcounter{enumi}{4}
\tightlist
\item
  Discussion:
\end{enumerate}

\begin{itemize}
\tightlist
\item
  P. 6: ``Previous research focused on impacts\ldots{}'' this sentence
  largely repeats a sentence from the introduction, not in the same
  words but in content (and using the same references). Please consider
  removing or modifying the sentence to add new insights for readers.
\end{itemize}

\textcolor{blue}{While we agree that this sentence is similar to one provided in the introduction, we believe it is important to restate this point here in order to provide context when discussing the contribution of our findings to existing literature assessing impacts of drought on agriculture. We therefore have not modified this sentence in our revised manuscript, as the purpose is to reinforce how our analysis differs from the existing state of methods and knowledge in the literature.}

\begin{itemize}
\tightlist
\item
  P. 6: ``\ldots{} demonstrate that the relationship between aquifer
  saturated thickness and drought impacts is highly nonlinear when
  accounting for farmers decisions about irrigated production areas.
  This indicates that sustainable groundwater conservation efforts
  {[}\ldots{]} should be targeted in space and time to avoid depletion
  exceeding critical thresholds \ldots{}'' Is the assumption of
  non-linearity tested? It is difficult to assess the validity of this
  statement based on the figures provided. Please clarify (e.g.~in the
  results) what supports this statement.
\end{itemize}

\textcolor{blue}{We now have statements that clarify this point in the last paragraph of section 2.3 on page 7 of the revised manuscript. Non-linearity in the relationship is visually inferred from examining the shape of the yield and production responses to combinations of weather (represented by water deficit) and aquifer (represented by saturated thickness) conditions. These relationships clearly demonstrate that responses are non-linear - for example increasing curvature of yield responses with higher water deficits, or the variable differences in yields and production when comparing different saturated thickness quantiles. Further formal statistical testing of non-linearity is beyond the scope of our paper given page and word limits, and given that non-linearities can be visually inferred as described above.}

\begin{enumerate}
\def\labelenumi{\arabic{enumi}.}
\setcounter{enumi}{5}
\tightlist
\item
  Methods - data sets:
\end{enumerate}

\begin{itemize}
\tightlist
\item
  P. 7: ``For counties that do not meet this condition, only rainfed
  yield observations are obtained to avoid the possibility that
  irrigated production may be only partly dependent on use of
  groundwater from the HPA.'' - Please consider whether this assumption
  could bias results and how potential biases can be ruled out.
\end{itemize}

\textcolor{blue}{
This will not bias the main conclusions from our the regression analysis, specifically the finding that lower saturated thickness reduces returns to irrigated crop production. Rainfed yields are included as part of the modeling merely to provide a comparison to the magnitude of yield impacts across saturated thickness categories for irrigated production. However, these rainfed yield data do not have any influence on the identification of irrigated yield response to water deficit for the three saturated thickness categories. Moreover, counties with only rainfed production observations are not included in regressions to predict irrigated area production shares, which focus solely on counties with both irrigated and rainfed production data in the same county. We have also added the following text on page 16 lines 352-355 of our revised manuscript to clarify this point:}

\textcolor{blue}{"Note that irrigated area share and average crop yield estimates use  data only from counties that have both irrigated and rainfed production observations. Counties with solely rainfed production observation data (Section 4.1) are excluded from these analyses, and are used solely for estimating impacts of drought on per-area crop yields (Equation 1)."}

\begin{itemize}
\tightlist
\item
  P. 9: Saturated thickness data - the number of monitoring wells would
  be of interest for readers to better judge the data, if available.
\end{itemize}

\textcolor{blue}{Aquifer thickness data were derived from interpolation of spatially dense water level observations from the United States Geological Survey's (USGS) national monitoring network. For the years of our analysis, aquifer thickness maps were generated based on between 14,000 and 20,000 annual point water level observations per year, distributed approximately evenly between counties conditional on variations in county area. While these data are a key input to our analysis, a detailed description of how saturated thickness data were generated is beyond the scope of our paper given the word limits imposed by articles in Nature Water. We refer the reader to  Haacker et al., for further information on the generation of saturated thickness data, which includes details about the number of well observations per year and quality control measures applied in production of saturated thickness maps. We have also added the following text to our revised manuscript (page 13 lines 291-293) to provide basic information on the number and distribution of water level observations used to construct aquifer thickness data:}

\textcolor{blue}{"Annual aquifer thickness maps were derived based on interpolation of 14,000 to 20,000 individual observations of water levels per year, along with detailed maps of both the land surface elevation and bedrock elevation of the HPA."}

\begin{enumerate}
\def\labelenumi{\arabic{enumi}.}
\setcounter{enumi}{6}
\tightlist
\item
  Methods - regression models:
\end{enumerate}

\begin{itemize}
\tightlist
\item
  P. 9: If the model includes fixed effects, how are the single plot
  lines for each category shown in Fig. 1 derived?
\end{itemize}

\textcolor{blue}{
That is because a single county-year was selected in predicting yields. We could pick any county-year without changing any of the conclusions we are drawing. This is because fixed effects are merely a shifter. The relative difference in yield across all the categories is the same no matter what county-year we pick, and the curvature will also be the same. Therefore, presenting lines for all the counties would unnecessarily complicate and crowd the figure without producing any more insights than a single estimated curve for a particular county.
}

\begin{itemize}
\tightlist
\item
  P. 9: Developments over time and independent variables: Please address
  early in the main text which overall trends shape developments in the
  aquifer studied. Please include key references about the case study
  region to inform this, such as Mrad et al.~(2020),
  doi.org/10.1073/pnas.2008383117. In light of this, please discuss the
  appropriateness of the approach. The model accounts for year-specific
  ``shocks'', but not other developments over time (e.g.~lagged effects,
  gradual depletion trends, the introduction of regulation, etc.).
  Relatedly, the analysis only saturated thickness and water deficits to
  explain yields, rather than adding more explanatory variables. The
  advantage of the approach taken here might be the simplicity and
  clarity of assumptions. Readers would benefit from a short explicit
  discussion.
\end{itemize}

\textcolor{blue}{Our analysis considers how aquifer depletion influences the resilience of irrigated crop production to drought risk. As part of this analysis, we explicitly consider trends in aquifer depletion over time through use of time-varying saturated thickness data for counties across the HPA. Consideration of other factors that might cause time variations in the exposure to drought risk are accounted for through the use of year and state fixed effects terms in our analysis. This allows us to isolate the effects of differences in saturated thickness on crop production responses to drought effects, which are the main focus of our paper. Explicit analysis of factors such as introduction of regulations is not feasible with available data, but some of these effects are embedded implicitly in our analysis through use of fixed effects terms. Moreover, our discussion includes detailed text (paragraph 5, section 3) outlining the need for future work to examine how adaptive responses by farmers and regulatory changes may alter relationships between aquifer thickness and drought risks, to which we have also added the suggested reference to support this discussion. Given word constraints and the focus of our paper, we are not in a position to provide a detailed background to the history of groundwater development in the HPA, for which we refer readers to the range of relevant references and supporting texts provided.}

\textcolor{blue}{One important omission from analysis, however, was variations in soil characteristics which might influence relationships between saturated thickness and drought risk at sub-state spatial scales. For example, if an area has sandier soils with lower water holding capacity then the impacts of lower saturated thickness (and therefore lower well yields) may be greater than a county dominated by soils with greater water holding capacities. As previously noted, we cannot include county fixed effects - an alternative way of accounting for soil type variability. This is because the majority of saturated thickness variation in our sample is cross-sectional rather than time-dependent, and hence using county-level fixed effects would remove the majority of saturated thickness variability from our analysis. Therefore, we re-ran our regressions this time including sand percentage, silt percentage, and water holding capacity from SSURGO alongside water deficit and saturated thickness predictors. Inclusion of soil properties did not change the conclusions or significance of our previously reported results.}

\textcolor{blue}{New references added:}

\textcolor{blue}{}

\begin{itemize}
\tightlist
\item
  P. 10: Please provide details on the basis functions in the methods
  section or the supplement.
\end{itemize}

\textcolor{blue}{We have added Appendix D in our revised manuscript, which provides a full overview of the basis functions used in our analysis.}

\begin{enumerate}
\def\labelenumi{\arabic{enumi}.}
\setcounter{enumi}{7}
\tightlist
\item
  Supplementary information: Please also consider providing additional
  information to give readers more clarity on the points outlined above
  in the supplementary information.
\end{enumerate}

\textcolor{blue}{Thank you for this suggestion. We have added a set of Appendices, which provide details about (A) the impact of water deficit on irrigated and average yield with confidence intervals, (B) statistical test of the impact of aquifer depletion on the share of irrigated acres relative to 130m aquifer thickness, (C) explanation of the reason for not including county fixed effects and tests of its potential implications for our results, and (D) how spline basis functions are used to estimate the relationship between two variables in a flexible manner without assuming any functional form. We believe that inclusion of these details have made our paper much stronger, and have referenced the appendices above in response to specific comments.}

\hypertarget{minor-points}{%
\subsection{MINOR POINTS:}\label{minor-points}}

\textcolor{blue}{Thank you for noticing there errors. All of these points have been addressed in the revised manuscript.}

\begin{itemize}
\item[$\boxtimes$]
  M1. Please add line numbers.
\item[$\boxtimes$]
  M2. P. 2: Missing closing parenthesis after ``(i.e., 3rd quantile
  \ldots{}''
\item[$\boxtimes$]
  M3. Fig. 1, 2, 3, and 5: Please add panel tags (``a'', ``b'' \ldots)
  to make it easier for readers to understand the captions.
\item[$\boxtimes$]
  M4. Fig. 3: Closing parenthesis of y-axis title missing
\item[$\boxtimes$]
  M5. P. 9: ``WB'' and ``ST'' should not be in italics, unless ``W'',
  ``B'', ``S'', and ``T'' are separate variables
\item[$\boxtimes$]
  M6. P. 9: ``{[}0 - 33.3\(\%\))j'' - the ``j'' does not seem to belong
  here
\item[$\boxtimes$]
  M7. P. 9: ``with sigma\_j represents the category specific intercept''
  should be ``with sigma\_j representing the category-specific
  intercept'' or something similar
\item[$\boxtimes$]
  M8. P. 9: ``under the generalize additive model framework'' should be
  ``generalized'' and ``modeling'' or something similar
\item[$\boxtimes$]
  M9. P. 9: What is ``mu\_t'' in equation 2? Should ``theta\_t'' in
  equation 1 be ``mu\_t''? Because ``theta\_jk'' is already the spline
  basis function.
\item[$\boxtimes$]
  M10. P. 9: One reference seems to be missing (the text includes
  ``(REF)'').
\end{itemize}

\end{document}
