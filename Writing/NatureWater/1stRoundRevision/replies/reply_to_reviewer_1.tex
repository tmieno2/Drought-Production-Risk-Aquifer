% Options for packages loaded elsewhere
\PassOptionsToPackage{unicode}{hyperref}
\PassOptionsToPackage{hyphens}{url}
%
\documentclass[
]{article}
\usepackage{amsmath,amssymb}
\usepackage{lmodern}
\usepackage{iftex}
\ifPDFTeX
  \usepackage[T1]{fontenc}
  \usepackage[utf8]{inputenc}
  \usepackage{textcomp} % provide euro and other symbols
\else % if luatex or xetex
  \usepackage{unicode-math}
  \defaultfontfeatures{Scale=MatchLowercase}
  \defaultfontfeatures[\rmfamily]{Ligatures=TeX,Scale=1}
\fi
% Use upquote if available, for straight quotes in verbatim environments
\IfFileExists{upquote.sty}{\usepackage{upquote}}{}
\IfFileExists{microtype.sty}{% use microtype if available
  \usepackage[]{microtype}
  \UseMicrotypeSet[protrusion]{basicmath} % disable protrusion for tt fonts
}{}
\makeatletter
\@ifundefined{KOMAClassName}{% if non-KOMA class
  \IfFileExists{parskip.sty}{%
    \usepackage{parskip}
  }{% else
    \setlength{\parindent}{0pt}
    \setlength{\parskip}{6pt plus 2pt minus 1pt}}
}{% if KOMA class
  \KOMAoptions{parskip=half}}
\makeatother
\usepackage{xcolor}
\usepackage[margin=1in]{geometry}
\usepackage{graphicx}
\makeatletter
\def\maxwidth{\ifdim\Gin@nat@width>\linewidth\linewidth\else\Gin@nat@width\fi}
\def\maxheight{\ifdim\Gin@nat@height>\textheight\textheight\else\Gin@nat@height\fi}
\makeatother
% Scale images if necessary, so that they will not overflow the page
% margins by default, and it is still possible to overwrite the defaults
% using explicit options in \includegraphics[width, height, ...]{}
\setkeys{Gin}{width=\maxwidth,height=\maxheight,keepaspectratio}
% Set default figure placement to htbp
\makeatletter
\def\fps@figure{htbp}
\makeatother
\setlength{\emergencystretch}{3em} % prevent overfull lines
\providecommand{\tightlist}{%
  \setlength{\itemsep}{0pt}\setlength{\parskip}{0pt}}
\setcounter{secnumdepth}{-\maxdimen} % remove section numbering
\ifLuaTeX
  \usepackage{selnolig}  % disable illegal ligatures
\fi
\IfFileExists{bookmark.sty}{\usepackage{bookmark}}{\usepackage{hyperref}}
\IfFileExists{xurl.sty}{\usepackage{xurl}}{} % add URL line breaks if available
\urlstyle{same} % disable monospaced font for URLs
\hypersetup{
  hidelinks,
  pdfcreator={LaTeX via pandoc}}

\author{}
\date{\vspace{-2.5em}}

\begin{document}

Reviewer \#1 (Comments for the Author):

This manuscript assesses the impact of aquifer thickness on agricultural
production during drought in the area overlying the High Plains aquifer
in the central United States. The authors use crop, meteorological, and
aquifer thickness data in regressions to develop relationships among
these quantities. Their most important finding is the impact of aquifer
thickness on productivity during drought and the ramifications for
aquifer conservation efforts. I have read this manuscript over a number
of times and found it of considerable interest and worthy of publication
in Nature Water. However, I do have a few concerns that I wanted to
bring to the attention of the authors. My most important concerns are as
follows (not necessarily in order of importance):

\textcolor{blue}{We really appreciate your comments. Addressing them has made our paper significantly stronger. Please find our replies to your comments written in blue.}

\begin{enumerate}
\def\labelenumi{\arabic{enumi}.}
\tightlist
\item
  Aquifer thickness versus saturated thickness -- The authors speak of
  ``aquifer saturated thickness'', which is a redundant term. Remember,
  the definition of an aquifer is a saturated unit -- there is no
  unsaturated aquifer. Please use ``aquifer thickness'' to avoid
  propagating this confusing usage into the future.
\end{enumerate}

\textcolor{blue}{All the instances of "saturated thickness" and "aquifer saturated thickness" have been replaced by "aquifer thickness".}

\begin{enumerate}
\def\labelenumi{\arabic{enumi}.}
\setcounter{enumi}{1}
\tightlist
\item
  Dependence on aquifer thickness -- I had a number of questions about
  the relationships displayed in Figure 1. First, the regression appears
  to have been done year by year. Are the quantiles recalculated each
  year to reflect that there may be changes between categories as time
  goes on? Second, when the authors state the lowest quantile goes from
  0 to 33\% are they assuming that the minimum thickness of 9 m is the
  zero point? Third, I assume that the county and year fixed effects are
  calculated as part of the regression. Are there any insights that can
  be gained from the interpretation of those quantities? Fourth, the
  authors need to point out that the same aquifer thickness can result
  in dramatically different transmissive characteristics. Thus, a
  relatively thin aquifer interval composed of coarse sands and gravels
  may be more resilient to drought than a much thicker aquifer dominated
  by fine sands and silts. Could that be a major player in the error
  bars in the plots? Please mention this at some point in the
  manuscript.
\end{enumerate}

\begin{itemize}
\tightlist
\item
  First point:
  \textcolor{blue}{Quantiles are not recalculated each each. If we do so, then the same quantile would represent different aquifer thickness range every year. When modeling the impact of aquifer thickness, this is not desirable because we are implicitly making assumptions that different ranges of aquifer thickness would have the same impact.}
\item
  Second point: \textbackslash textcolor\{blue\}\{Actually, 0\% quntile
  is 12 m. ``9m'' was repalced with ``12m'' in the current manuscript.\}
\item
  Third point:
  \textcolor{blue}{No interesting insights will be gained from them because they are merely shifters without any intractions with aquifer thickness. They tell us how much higher yield in a single county relative to another county on average. But, we do not beleive that is useful information at least in this artcile.}
\item
  Fourth point:
  \textcolor{blue}{This is a very important point. We have added texts describing the implications of this fact in page xx.}
\end{itemize}

\begin{enumerate}
\def\labelenumi{\arabic{enumi}.}
\setcounter{enumi}{2}
\tightlist
\item
  Conditioned on drought risk exposure -- I wasn't sure what the authors
  meant by this in the second paragraph of Section 2.2. This must be
  related to the water deficit but the lack of explanation in the main
  text or the methods section left me wondering. Please clarify at some
  point in the manuscript.
\end{enumerate}

\textcolor{blue}{What we meant by "conditiond on drought risk exposure" is "after taking into account the impact of climate variables." In order to avoid confusion, we simply removed it from the sentence in the current manuscript.}

\begin{enumerate}
\def\labelenumi{\arabic{enumi}.}
\setcounter{enumi}{3}
\tightlist
\item
  Equation 2 -- Please define all terms in the equation and provide an
  explanation of what is meant by ``estimated semi-parametrically''.
\end{enumerate}

\textcolor{blue}{We now have a more detailed mathematical expressions of the model with all the terms explained. We eliminated the word semi-parametrically as it is not ncecessary. Generalized additive model that estiamtes the impact of a variable using splines without functional form assumption are sometimes referred to semi-parametric regression. That's what we meant by "estimate semi-parametrically" in our previous manuscript.}

\begin{enumerate}
\def\labelenumi{\arabic{enumi}.}
\setcounter{enumi}{4}
\tightlist
\item
  Kansas and California efforts -- I've seen a number of papers
  describing efforts in Kansas and California to address aquifer
  depletion via LEMAs (Kansas) and SGMA (California). The 1985-2016
  period examined by the authors was before most of the efforts in
  Kansas got started. My understanding is that the first LEMA was
  established in 2013 but then no more were set up until after 2016.
  This manuscript would be more impactful if the authors could speculate
  on how their relationships might change with greater attention to
  groundwater conservation. For example, I would think the decrease in
  yield the authors observe with negative water deficits would disappear
  when the irrigators are using soil moisture sensors and are focused on
  water-use efficiency. I would expect changes as well for tipping
  points.
\end{enumerate}

\textcolor{blue}{Tim?}

\hypertarget{minor-points}{%
\subsection{Minor points}\label{minor-points}}

\begin{itemize}
\tightlist
\item
  Section 1, 3rd Paragraph -- ``Two aquifer pathways exist through
  aquifer depletion will..'' -- confusing sentence -- please replace
  ``through'' with ``by which''.
\end{itemize}

\textcolor{blue}{Corrected.}

\begin{itemize}
\tightlist
\item
  Section 1, 4th Paragraph -- ``we generate empirical evidence'' seems
  awkward to me. How about replacing with ``we present empirical
  evidence''? Next sentence -- ``Our analysis advances on
  previous\ldots{}'' reads better as ``Our analysis extends
  previous\ldots{}'' There are many issues such as these and the
  previous comment. Please address such issues so that the manuscript is
  an easier read.
\end{itemize}

\textcolor{blue}{We completely agree. Both of the suggested changes have been made.}

\begin{itemize}
\tightlist
\item
  Section 2.1 title -- I would delete ``marginally'' as it is not needed
  and some will find it confusing.
\end{itemize}

\textcolor{blue}{Tim, any thoughts on this. I feel like it is still important to say the increase is not much.}

\begin{itemize}
\tightlist
\item
  Figure 2 - Why do the plots stop at a thickness of about 12 m when you
  had earlier defined the minimum thickness as 9 m and mention 10 m in
  the text when discussing the relationships?
\end{itemize}

\textcolor{blue}{Thank you for catching this. 9m and 10m should have been 12m.}

\begin{itemize}
\tightlist
\item
  Section 2.3, First Paragraph -- please clarify that ``average per-area
  production'' is a combination of irrigated and rainfed yields. That
  would make the manuscript easier to follow.
\end{itemize}

\textcolor{blue}{We added "(weighted-average of irrigated and rainfed yields, where weights are production area shares)" so that it is clear what average per-area production means.}

\begin{itemize}
\tightlist
\item
  Figure 5 -- I assume that the left plot is corn and the right soybean,
  but please clarify.
\end{itemize}

\textcolor{blue}{Thanks for noticing this. Fixed.}

\begin{itemize}
\tightlist
\item
  Data and Code Availability -- Are the data from Dr.~Haacker on the
  Github repository? Please place them there prior to publication so
  future investigators can access them rather than having to track down
  Dr.~Haacker. The data set is over seven years old so it should be made
  available according to the principles of open science.
\end{itemize}

\textcolor{blue}{Yes, we will make the data available online.}

\begin{itemize}
\tightlist
\item
  Line before Equation (2) -- ``REF'' should be replaced with the actual
  reference.
\end{itemize}

\textcolor{blue}{Tim?}

\begin{itemize}
\tightlist
\item
  Discussion, First Paragraph -- ``reduction in aquifer depletion''
  should be ``reduction in aquifer thickness''.
\end{itemize}

\textcolor{blue}{Thank you for catching this. Fixed.}

\begin{itemize}
\tightlist
\item
  Section 4.2, 3rd Paragraph -- ``the the impact'' -- remember to spell
  and grammar check prior to submission.
\end{itemize}

\textcolor{blue}{Thank you for catching this. Fixed.}

\begin{itemize}
\tightlist
\item
  Reference in Section 3 -- I doubt that ``Butler Jr'' is the surname,
  please replace with ``Butler''. Suffixes apparently aren't handled
  well by some of the bibliography generation programs.
\end{itemize}

\textcolor{blue}{}

\begin{itemize}
\tightlist
\item
  Please remember to close all parentheses and don't mispair brackets
  and parentheses. Such stuff can raise questions about the care used in
  the analysis.
\end{itemize}

\end{document}
