% Options for packages loaded elsewhere
\PassOptionsToPackage{unicode}{hyperref}
\PassOptionsToPackage{hyphens}{url}
%
\documentclass[
]{article}
\usepackage{amsmath,amssymb}
\usepackage{lmodern}
\usepackage{iftex}
\ifPDFTeX
  \usepackage[T1]{fontenc}
  \usepackage[utf8]{inputenc}
  \usepackage{textcomp} % provide euro and other symbols
\else % if luatex or xetex
  \usepackage{unicode-math}
  \defaultfontfeatures{Scale=MatchLowercase}
  \defaultfontfeatures[\rmfamily]{Ligatures=TeX,Scale=1}
\fi
% Use upquote if available, for straight quotes in verbatim environments
\IfFileExists{upquote.sty}{\usepackage{upquote}}{}
\IfFileExists{microtype.sty}{% use microtype if available
  \usepackage[]{microtype}
  \UseMicrotypeSet[protrusion]{basicmath} % disable protrusion for tt fonts
}{}
\makeatletter
\@ifundefined{KOMAClassName}{% if non-KOMA class
  \IfFileExists{parskip.sty}{%
    \usepackage{parskip}
  }{% else
    \setlength{\parindent}{0pt}
    \setlength{\parskip}{6pt plus 2pt minus 1pt}}
}{% if KOMA class
  \KOMAoptions{parskip=half}}
\makeatother
\usepackage{xcolor}
\usepackage[margin=1in]{geometry}
\usepackage{graphicx}
\makeatletter
\def\maxwidth{\ifdim\Gin@nat@width>\linewidth\linewidth\else\Gin@nat@width\fi}
\def\maxheight{\ifdim\Gin@nat@height>\textheight\textheight\else\Gin@nat@height\fi}
\makeatother
% Scale images if necessary, so that they will not overflow the page
% margins by default, and it is still possible to overwrite the defaults
% using explicit options in \includegraphics[width, height, ...]{}
\setkeys{Gin}{width=\maxwidth,height=\maxheight,keepaspectratio}
% Set default figure placement to htbp
\makeatletter
\def\fps@figure{htbp}
\makeatother
\setlength{\emergencystretch}{3em} % prevent overfull lines
\providecommand{\tightlist}{%
  \setlength{\itemsep}{0pt}\setlength{\parskip}{0pt}}
\setcounter{secnumdepth}{-\maxdimen} % remove section numbering
\ifLuaTeX
  \usepackage{selnolig}  % disable illegal ligatures
\fi
\IfFileExists{bookmark.sty}{\usepackage{bookmark}}{\usepackage{hyperref}}
\IfFileExists{xurl.sty}{\usepackage{xurl}}{} % add URL line breaks if available
\urlstyle{same} % disable monospaced font for URLs
\hypersetup{
  hidelinks,
  pdfcreator={LaTeX via pandoc}}

\author{}
\date{\vspace{-2.5em}}

\begin{document}

Reviewer \#1 (Comments for the Author):

This manuscript assesses the impact of aquifer thickness on agricultural
production during drought in the area overlying the High Plains aquifer
in the central United States. The authors use crop, meteorological, and
aquifer thickness data in regressions to develop relationships among
these quantities. Their most important finding is the impact of aquifer
thickness on productivity during drought and the ramifications for
aquifer conservation efforts. I have read this manuscript over a number
of times and found it of considerable interest and worthy of publication
in Nature Water. However, I do have a few concerns that I wanted to
bring to the attention of the authors. My most important concerns are as
follows (not necessarily in order of importance):

\textcolor{blue}{Thank you for the detailed and constructive feedback on our manuscript. We have endeavored to address all your comments and suggestions, and our replies to your comments can found in blue below (original comments in black)}

\begin{enumerate}
\def\labelenumi{\arabic{enumi}.}
\tightlist
\item
  Aquifer thickness versus saturated thickness -- The authors speak of
  \texttt{aquifer\ saturated\ thickness\textquotesingle{}\textquotesingle{},\ which\ is\ a\ redundant\ term.\ Remember,\ the\ definition\ of\ an\ aquifer\ is\ a\ saturated\ unit\ –\ there\ is\ no\ unsaturated\ aquifer.\ Please\ use}aquifer
  thickness'\,' to avoid propagating this confusing usage into the
  future.
\end{enumerate}

\textcolor{blue}{We agree that there is some redundant use of words here. All the instances of ``saturated thickness'' and ``aquifer saturated thickness'' have been replaced by ``aquifer thickness'' in the revised manuscript.}

\begin{enumerate}
\def\labelenumi{\arabic{enumi}.}
\setcounter{enumi}{1}
\tightlist
\item
  Dependence on aquifer thickness -- I had a number of questions about
  the relationships displayed in Figure 1. First, the regression appears
  to have been done year by year. Are the quantiles recalculated each
  year to reflect that there may be changes between categories as time
  goes on? Second, when the authors state the lowest quantile goes from
  0 to 33\% are they assuming that the minimum thickness of 9 m is the
  zero point? Third, I assume that the county and year fixed effects are
  calculated as part of the regression. Are there any insights that can
  be gained from the interpretation of those quantities? Fourth, the
  authors need to point out that the same aquifer thickness can result
  in dramatically different transmissive characteristics. Thus, a
  relatively thin aquifer interval composed of coarse sands and gravels
  may be more resilient to drought than a much thicker aquifer dominated
  by fine sands and silts. Could that be a major player in the error
  bars in the plots? Please mention this at some point in the
  manuscript.
\end{enumerate}

\begin{itemize}
\item
  First point:
  \textcolor{blue}{Quantiles are not recalculated each year. If we do so, then the same quantile would represent different aquifer thickness range every year. When modeling the impact of aquifer thickness, this is not desirable because it would be expected that different ranges of aquifer thickness would have the same impact on a farmer's ability to buffer drought risk irrespective of year.}
\item
  Second point:
  \textcolor{blue}{Thank you for highlighting this point. For the 0\% quantile, we assume a saturated thickness of 9m. This minimum of saturated thickness was selected based on \textcolor{red}{Tim, please add REFs}, which defines 9m as the minimum aquifer thickness needed to provide a functioning water supply. We have modified the text on page X lines X-X of the revised manuscript to note these points.}
\item
  Third point:
  \textcolor{blue}{Yes, country and year fixed effects are calculated as part of the regressions. However, these variables can be thought of simply as shifters that tell us how much higher average yields are in a single county relative to another county and to account for trends in yields over time. As these variables do not interact with aquifer thickness in our analysis, there would be little insights gained from analyzing these variables - and indeed it is standard to not report and analyze fixed effects terms in econometric regressions used in statistical crop-climate modeling.}
\item
  Fourth point:
  \textcolor{blue}{This is an important point, which we thank the reviewer for highlighting. The reviewer is correct that unobserved factors, such as aquifer heterogeneity influencing well yield response to changes in saturated thickness, are a likely underlying cause of the uncertainty ranges found in our analyses. Variations in aquifer conductivity would alter the relationship between saturated thickness and well capacity, thereby adding noise to the relationship between saturated thickness and drought risk exposure that would contribute to uncertainty in our estimates of observed relationships between drought impacts and aquifer conditions. \textcolor{red}{(Tim, I am adding this sentence.) This noise is referred to as measurement error in the field of statistics and econometrics, and it has a well-known consequence in regression analysis, called attenuation bias. Specifically, measurement error in an explanatory variable (here, aquifer thickness) tends to bias your estimation in a way that makes the impact of the explanatory variable with measurement errors appear "less" influential than it true is. So, in our context, our estimated impact of aquifer thickness on the share of irrigation is conservative. In reality, the impact of aquifer thickness is likely to be larger.} Importantly though, our analysis shows that - even in the presence of these uncertainties - we are still able to identify statistically significant increases in drought risk exposure for irrigated agriculture as a function of declining saturated thickness. This further reinforces the importance of aquifer conditions as a determinant of adaptive capacity to drought. To highlight these points, we have added the following text in the Discussion section on page 10, the 4th paragraph of our revised manuscript:}
\end{itemize}

\textcolor{blue}{``Similarly, unobserved variations in aquifer properties will also introduce noise into the relationships between aquifer thickness, well yields, and drought risk exposure. For example, a thinner aquifer interval composed of coarse sands and gravels may be able to supply comparable or higher well yields than a thicker aquifer dominated by fine sands and silt deposits, and hence offer greater resilience to drought despite similar or larger aquifer thickness [Butler et al., 2013, Korus and Hensen, 2020]. . These uncertainties can be considered as measurement errors in aquifer thickness as a proxy for well yield. A well-known consequence of measurement errors in regression analysis is attenuation bias [Bound and Krueger, 1991, Hyslop and Imbens, 2001]. That is, measurement error in an explanatory variable (here, aquifer thickness) tends to bias your estimation in a way that makes the impact of the explanatory variable with measurement errors appear "less" influential than it true is. Therefore, it is likely that our estimated impacts of aquifer thickness on drought risk is rather conservative, and the true impact of groundwater depletion is likely to be greater than what we reported in this article. Importantly, however, our analysis shows that - even in the presence of these uncertainties - we are still able to identify statistically significant increases in drought risk exposure for irrigated agriculture as a function of declining saturated thickness. Nonetheless, improved data on spatial variations in aquifer properties and greater monitoring of real-world changes in well capacities alongside traditional water level measurements would help to refine understanding of patterns of drought vulnerability across the HPA. This would enable more precise estimates to be made of critical tipping points in the resilience of irrigated agriculture to drought, which are needed to effectively design and target groundwater conservation policies.''}

\textcolor{blue}{New references added:}

\begin{itemize}
\item
  \textcolor{blue}{Butler, J. J., Stotler, R. L., Whittemore, D. O., and Reboulet, E. C. (2013). Interpretation of water level changes in the High Plains aquifer in western Kansas. Groundwater, 51(2), 180-190.}
\item
  \textcolor{blue}{Korus, J. T., and Hensen, H. J. (2020). Depletion percentage and nonlinear transmissivity as design criteria for groundwater-level observation networks. Environmental Earth Sciences, 79(16), 382.}
\item
  \textcolor{blue}{John Bound and Alan B Krueger. The extent of measurement error in longitudinal earnings data: Do two wrongs make a right? Journal of labor economics, 9(1):1–24, 1991.}
\item
  \textcolor{blue}{Dean R Hyslop and Guido W Imbens. Bias from classical and other forms of measurement error. Journal of Business \& Economic Statistics, 19(4):475–481, 2001}
\end{itemize}

\begin{enumerate}
\def\labelenumi{\arabic{enumi}.}
\setcounter{enumi}{2}
\tightlist
\item
  Conditioned on drought risk exposure -- I wasn't sure what the authors
  meant by this in the second paragraph of Section 2.2. This must be
  related to the water deficit but the lack of explanation in the main
  text or the methods section left me wondering. Please clarify at some
  point in the manuscript.
\end{enumerate}

\textcolor{blue}{By ``conditioned on drought risk exposure'', we were referring to the fact that Figure 4 displays the relationship between saturated thickness and the share of production area that is irrigated while also accounting for variations in climate across the aquifer. I.e. that the Figure shows the impact of saturated thickness on share of production irrigated after removing the effects of climate on share of production irrigated. To improve clarity, we have modified this phrase to now state ``...after accounting for spatio-temporal variations in drought risk exposure across the aquifer''.}

\begin{enumerate}
\def\labelenumi{\arabic{enumi}.}
\setcounter{enumi}{3}
\tightlist
\item
  Equation 2 -- Please define all terms in the equation and provide an
  explanation of what is meant by ``estimated semi-parametrically'\,'.
\end{enumerate}

\textcolor{blue}{We now have a more detailed mathematical expressions of the model with all the terms explained in Section 4.2 of our revised manuscript. The word semi-parametrically refers to the fact that generalized additive models, such as used in our analysis, estimate the impact of a variable using splines without functional form assumption. These are sometimes referred to as semi-parametric regression methods. However, we agree that the term is potentially confusing, and therefore have removed the term ``semi-parametrically'' in our revised manuscript.}

\begin{enumerate}
\def\labelenumi{\arabic{enumi}.}
\setcounter{enumi}{4}
\tightlist
\item
  Kansas and California efforts -- I've seen a number of papers
  describing efforts in Kansas and California to address aquifer
  depletion via LEMAs (Kansas) and SGMA (California). The 1985-2016
  period examined by the authors was before most of the efforts in
  Kansas got started. My understanding is that the first LEMA was
  established in 2013 but then no more were set up until after 2016.
  This manuscript would be more impactful if the authors could speculate
  on how their relationships might change with greater attention to
  groundwater conservation. For example, I would think the decrease in
  yield the authors observe with negative water deficits would disappear
  when the irrigators are using soil moisture sensors and are focused on
  water-use efficiency. I would expect changes as well for tipping
  points.
\end{enumerate}

\textcolor{blue}{This is an interesting idea - thank you for raising it. We agree that efforts to improve aquifer management could help to mitigate some of the changes in productivity driven by climate and aquifer conditions in our analysis. Greater attention to groundwater conservation, and use of improved scheduling technologies such as soil moisture probes, would be expected to mitigate production losses in years with negative water deficits (i.e. wetter years) which we attribute to increased possibility for over-irrigation when well capacities are high (and hence when farmers do not face or observe physical scarcity). In years with large positive water deficits (i.e. drought years), improved irrigation water management would also be expected to mitigate some of the additional production risk caused by lower saturated thickness. However, we would still expect lower saturated thickness to result in elevated exposure to drought risk even in these circumstances, as declining well yields would create a physical limit to water supply which would stop farmers being able to fully meet crop water needs in these years. Formally testing these effects is beyond the scope and data of our study, but we agree that these are valuable areas for future work and have added the following text to the discussion of our revised manuscript on page X lines X-X:}

\textcolor{blue}{``Improvements to irrigation scheduling and water management practices, such as have been implemented as part of the SD6-LEMA (Local Enhanced Management Area) in Kansas (Deines et al., 2019; Glose et al., 2022) and proposed as part of California's SGMA (Sustainable Groundwater Management Act) (Berbel and Esteban, 2019; Lubell et al., 2020), could also play an important role in improving the efficiency of irrigation and, hence, the ability to more effectively meet crop water demands with lower well capacities. These kinds of interventions would be likely to reduce - but not eliminate - the increases in drought risk caused by declining aquifer saturated thickness, and could bring additional benefits of reducing over-use of water in wetter years through greater awareness of efficiency saving potentials (Foster et al., 2019) and the importance of conservation (Marston et al., 2022) amongst growers.''}

\textcolor{blue}{New references added:}

\begin{itemize}
\item
  \textcolor{blue}{Glose, T. J., Zipper, S., Hyndman, D. W., Kendall, A. D., Deines, J. M., and Butler, J. J. (2022). Quantifying the impact of lagged hydrological responses on the effectiveness of groundwater conservation. Water Resources Research, 58(7), e2022WR032295.}
\item
  \textcolor{blue}{Berbel, J., and Esteban, E. (2019). Droughts as a catalyst for water policy change. Analysis of Spain, Australia (MDB), and California. Global Environmental Change, 58, 101969.}
\item
  \textcolor{blue}{Lubell, M., Blomquist, W., and Beutler, L. (2020). Sustainable groundwater management in California: A grand experiment in environmental governance. Society and Natural Resources, 33(12), 1447-1467.}
\item
  \textcolor{blue}{Marston, L. T., Zipper, S., Smith, S. M., Allen, J. J., Butler, J. J., Gautam, S., and David, J. Y. (2022). The importance of fit in groundwater self-governance. Environmental Research Letters, 17(11), 111001.}
\end{itemize}

\hypertarget{minor-points}{%
\subsection{Minor points}\label{minor-points}}

\begin{itemize}
\tightlist
\item
  Section 1, 3rd Paragraph --
  \texttt{Two\ aquifer\ pathways\ exist\ through\ aquifer\ depletion\ will..\textquotesingle{}\textquotesingle{}\ –\ confusing\ sentence\ –\ please\ replace}through'\,'
  with ``by which'\,'.
\end{itemize}

\textcolor{blue}{Corrected.}

\begin{itemize}
\tightlist
\item
  Section 1, 4th Paragraph --
  \texttt{we\ generate\ empirical\ evidence\textquotesingle{}\textquotesingle{}\ seems\ awkward\ to\ me.\ How\ about\ replacing\ with}we
  present empirical evidence'`? Next sentence --
  \texttt{Our\ analysis\ advances\ on\ previous\ \$\textbackslash{}dots\$\textquotesingle{}\textquotesingle{}\ reads\ better\ as}Our
  analysis extends previous \(\dots\)'' There are many issues such as
  these and the previous comment. Please address such issues so that the
  manuscript is an easier read.
\end{itemize}

\textcolor{blue}{We completely agree. Both of the suggested changes have been made.}

\begin{itemize}
\tightlist
\item
  Section 2.1 title -- I would delete ``marginally'\,' as it is not
  needed and some will find it confusing.
\end{itemize}

\textcolor{blue}{Tim, any thoughts on this. I feel like it is still important to say the increase is not much.}

\begin{itemize}
\tightlist
\item
  Figure 2 - Why do the plots stop at a thickness of about 12 m when you
  had earlier defined the minimum thickness as 9 m and mention 10 m in
  the text when discussing the relationships?
\end{itemize}

\textcolor{blue}{Thank you for catching this. 9m and 10m should have been 12m.}

\begin{itemize}
\tightlist
\item
  Section 2.3, First Paragraph -- please clarify that ``average per-area
  production'\,' is a combination of irrigated and rainfed yields. That
  would make the manuscript easier to follow.
\end{itemize}

\textcolor{blue}{We added ``weighted-average of irrigated and rainfed yields, where weights are production area shares'' so that it is clear what average per-area production means.}

\begin{itemize}
\tightlist
\item
  Figure 5 -- I assume that the left plot is corn and the right soybean,
  but please clarify.
\end{itemize}

\textcolor{blue}{Thanks for noticing this. Fixed.}

\begin{itemize}
\tightlist
\item
  Data and Code Availability -- Are the data from Dr.~Haacker on the
  Github repository? Please place them there prior to publication so
  future investigators can access them rather than having to track down
  Dr.~Haacker. The data set is over seven years old so it should be made
  available according to the principles of open science.
\end{itemize}

\textcolor{blue}{Yes, we will make the data available online.}

\begin{itemize}
\tightlist
\item
  Line before Equation (2) -- ``REF'\,' should be replaced with the
  actual reference.
\end{itemize}

\textcolor{blue}{Tim?}

\begin{itemize}
\tightlist
\item
  Discussion, First Paragraph --
  \texttt{reduction\ in\ aquifer\ depletion\textquotesingle{}\textquotesingle{}\ should\ be}reduction
  in aquifer thickness'\,'.
\end{itemize}

\textcolor{blue}{Thank you for catching this. Fixed.}

\begin{itemize}
\tightlist
\item
  Section 4.2, 3rd Paragraph -- ``the the impact'\,' -- remember to
  spell and grammar check prior to submission.
\end{itemize}

\textcolor{blue}{Thank you for catching this. Fixed.}

\begin{itemize}
\tightlist
\item
  Reference in Section 3 -- I doubt that
  \texttt{Butler\ Jr\textquotesingle{}\textquotesingle{}\ is\ the\ surname,\ please\ replace\ with}Butler'\,'.
  Suffixes apparently aren't handled well by some of the bibliography
  generation programs.
\end{itemize}

\textcolor{blue}{}

\begin{itemize}
\tightlist
\item
  Please remember to close all parentheses and don't mispair brackets
  and parentheses. Such stuff can raise questions about the care used in
  the analysis.
\end{itemize}

\end{document}
