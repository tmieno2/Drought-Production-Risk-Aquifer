% Options for packages loaded elsewhere
\PassOptionsToPackage{unicode}{hyperref}
\PassOptionsToPackage{hyphens}{url}
%
\documentclass[
]{article}
\usepackage{amsmath,amssymb}
\usepackage{lmodern}
\usepackage{iftex}
\ifPDFTeX
  \usepackage[T1]{fontenc}
  \usepackage[utf8]{inputenc}
  \usepackage{textcomp} % provide euro and other symbols
\else % if luatex or xetex
  \usepackage{unicode-math}
  \defaultfontfeatures{Scale=MatchLowercase}
  \defaultfontfeatures[\rmfamily]{Ligatures=TeX,Scale=1}
\fi
% Use upquote if available, for straight quotes in verbatim environments
\IfFileExists{upquote.sty}{\usepackage{upquote}}{}
\IfFileExists{microtype.sty}{% use microtype if available
  \usepackage[]{microtype}
  \UseMicrotypeSet[protrusion]{basicmath} % disable protrusion for tt fonts
}{}
\makeatletter
\@ifundefined{KOMAClassName}{% if non-KOMA class
  \IfFileExists{parskip.sty}{%
    \usepackage{parskip}
  }{% else
    \setlength{\parindent}{0pt}
    \setlength{\parskip}{6pt plus 2pt minus 1pt}}
}{% if KOMA class
  \KOMAoptions{parskip=half}}
\makeatother
\usepackage{xcolor}
\usepackage[margin=1in]{geometry}
\usepackage{graphicx}
\makeatletter
\def\maxwidth{\ifdim\Gin@nat@width>\linewidth\linewidth\else\Gin@nat@width\fi}
\def\maxheight{\ifdim\Gin@nat@height>\textheight\textheight\else\Gin@nat@height\fi}
\makeatother
% Scale images if necessary, so that they will not overflow the page
% margins by default, and it is still possible to overwrite the defaults
% using explicit options in \includegraphics[width, height, ...]{}
\setkeys{Gin}{width=\maxwidth,height=\maxheight,keepaspectratio}
% Set default figure placement to htbp
\makeatletter
\def\fps@figure{htbp}
\makeatother
\setlength{\emergencystretch}{3em} % prevent overfull lines
\providecommand{\tightlist}{%
  \setlength{\itemsep}{0pt}\setlength{\parskip}{0pt}}
\setcounter{secnumdepth}{-\maxdimen} % remove section numbering
\ifLuaTeX
  \usepackage{selnolig}  % disable illegal ligatures
\fi
\IfFileExists{bookmark.sty}{\usepackage{bookmark}}{\usepackage{hyperref}}
\IfFileExists{xurl.sty}{\usepackage{xurl}}{} % add URL line breaks if available
\urlstyle{same} % disable monospaced font for URLs
\hypersetup{
  hidelinks,
  pdfcreator={LaTeX via pandoc}}

\author{}
\date{\vspace{-2.5em}}

\begin{document}

\textcolor{blue}{Thank you for the detailed and constructive feedback on our manuscript. We have endeavored to address all your comments and suggestions, and our replies to your comments can found in blue below (original comments in black).}

This is an interesting paper that asks an important yet answered
question in the field of agricultural sustainability - how much is
groundwater depletion reducing agricultural productivity in the High
Plains region in the United States? While the paper is an important
contribution to the literature, there are several issues that need to be
addressed prior to acceptance at any journal.

\begin{enumerate}
\def\labelenumi{\arabic{enumi}.}
\tightlist
\item
  There needs to be more description about why aquifer saturation
  thickness is an appropriate proxy for farmers' ability to irrigate
  from wells. This is particularly true since this is a broad interest
  journal, and the mechanism for the main independent variable should be
  very clear to all readers.
\end{enumerate}

\textcolor{blue}{Saturated thickness affects farmers ability to meet the water requirements of their crops due to its impacts on the pumping yields (or capacities) of groundwater irrigation wells. As the saturated thickness of an aquifer declines, the rate at which water can flow from the aquifer to the pumping well is reduced proportionally to the aquifer's transmissivity (equal to the hydraulic conductivity of the aquifer geology multiplied by the saturated thickness of that aquifer geology). For small relative changes in saturated thickness (e.g. for initial depletion of an aquifer with large initial saturated thickness), the change in well yield is small and unlikely to have substantial negative impacts on drought risk exposure as a farmer could simply run their pump for slightly longer to compensate for the small reduction in instantaneous well yield. However, for larger relative changes in saturated thickness (e.g. due to significant aquifer depletion and/or thinner aquifer bodies), the magnitude of well yield reductions will be much larger. These reductions in water supply capacity cannot be mitigated by changes to irrigation scheduling or management practices. The farmer will often be unable to keep up with crop water demands during the peak of the growing season in the absence of rainfall, even if running their system continually. This effect can also typically not be mitigated by drilling additional or deeper wells, due to a combination of both regulatory moratoriums on well drilling - which are common across the HPA - and high costs of deep well construction. To more clearly highlight for non-specialist readers these connections between saturated thickness, well yields, and drought resilience of agriculture, we have added the following text on page X lines X-X of our revised manuscript. We note that we are limited in elaborating further on background information given journal word constraints, and hence provide key supporting studies in citations for readers interested in deepening understanding of these dynamics.}

\textcolor{blue}{``...depletion also reduces the saturated thickness of an aquifer, lowering the aquifer's transmissivity (i.e., the potential rate of water supply to wells). This latter effect reduces the rate at which water can be pumped and applied to a field as irrigation (Konikow and Kendy, 2005; Foster et al., 2015; Hrozencik et al., 2017). If the reduction in well yields is substantial - for example as would occur when a large proportion of aquifer saturated thickness is depleted (Hecox et al., 2002; Korus and Hensen, 2020) - then a farmer may be unable to extract sufficient water to meet crop water demands, in particular during periods of limited rainfall and in the peak of the growing season when crops are most sensitive to water deficits (Foster et al., 2015a; Rouhi-Rad et al., 2020).''
}

\textcolor{blue}{Additional references added:}

\begin{itemize}
\item
  \textcolor{blue}{Hecox, G.R., Macfarlane, P.A., and Wilson B.B. (2002). Calculation of yield for High Plains wells: relationship between saturated thickness and well yield. Open file report 2002–25C, Kansas Geological Survey: Lawrence, Kansas, USA.}
\item
  \textcolor{blue}{Korus, J. T., and Hensen, H. J. (2020). Depletion percentage and nonlinear transmissivity as design criteria for groundwater-level observation networks. Environmental Earth Sciences, 79(16), 382.}
\item
  \textcolor{blue}{Rouhi-Rad, M., Araya, A., and Zambreski, Z. T. (2020). Downside risk of aquifer depletion. Irrigation Science, 38, 577-591.}
\end{itemize}

\begin{enumerate}
\def\labelenumi{\arabic{enumi}.}
\setcounter{enumi}{1}
\tightlist
\item
  Though I am not trained as a hydrologist, my understanding is that
  aquifer saturation thickness is a good measure of general water
  availability in an aquifer but not necessarily how much water is
  available at a specific location or point. If this is the case, the
  authors should make clear why this is an appropriate measure to use in
  their study and potential limitations of using this approach.
\end{enumerate}

\textcolor{blue}{This is a great point, which is also pointed out by another reviewer. In response, we have added the following texts in the Discussion section on page 10, the 4th paragraph of our revised manuscript.}

\textcolor{blue}{``Similarly, unobserved variations in aquifer properties will also introduce noise into the relationships between aquifer thickness, well yields, and drought risk exposure. For example, a thinner aquifer interval composed of coarse sands and gravels may be able to supply comparable or higher well yields than a thicker aquifer dominated by fine sands and silt deposits, and hence offer greater resilience to drought despite similar or larger aquifer thickness [Butler et al., 2013, Korus and Hensen, 2020]. . These uncertainties can be considered as measurement errors in aquifer thickness as a proxy for well yield. A well-known consequence of measurement errors in regression analysis is attenuation bias [Bound and Krueger, 1991, Hyslop and Imbens, 2001]. That is, measurement error in an explanatory variable (here, aquifer thickness) tends to bias your estimation in a way that makes the impact of the explanatory variable with measurement errors appear "less" influential than it true is. Therefore, it is likely that our estimated impacts of aquifer thickness on drought risk is rather conservative, and the true impact of groundwater depletion is likely to be greater than what we reported in this article. Importantly, however, our analysis shows that - even in the presence of these uncertainties - we are still able to identify statistically significant increases in drought risk exposure for irrigated agriculture as a function of declining saturated thickness. Nonetheless, improved data on spatial variations in aquifer properties and greater monitoring of real-world changes in well capacities alongside traditional water level measurements would help to refine understanding of patterns of drought vulnerability across the HPA. This would enable more precise estimates to be made of critical tipping points in the resilience of irrigated agriculture to drought, which are needed to effectively design and target groundwater conservation policies.''}

\begin{enumerate}
\def\labelenumi{\arabic{enumi}.}
\setcounter{enumi}{2}
\tightlist
\item
  It is unclear to me why county and time fixed effects were not used in
  every regression. In particular, it does not look like they were
  included in the regression that modeled irrigated area as a function
  of saturated thickness and water deficit. Yet, accounting for
  time-invariant differences across counties is important, especially
  because this is the regression that showed the largest effect of
  saturated thickness on a measure of agricultural production (irrigated
  area). Is there not enough variation within a county through time to
  pick up the effect of saturated thickness on irrigated area? This
  should all be made clearer, and a model that better captures the
  causal relationship between saturated thickness and area irrigated
  should be used if possible. If it's not possible, the limitations of
  the current approach should be made very clear and potential issues
  with inference.
\end{enumerate}

\textcolor{blue}{Thank you for this comment, which we believe led to a slightly better model and acknowledging important limitations of our methods. County and year fixed effects were included in equation (1), but only year fixed effects were included in equation (2). As you guessed, the problem of including county FEs lies in the massive elimination of variation in aquifer thickness. Please see Figure 6 in Appendix A for the histogram of (1) aquifer thickness demeaned by the county average, (2) aquifer thickness demeaned by the state average, and (3) aquifer thickness without demeaning. As you can see in the figure, inclusion of county FEs (mathematically equivalent to regression using county-demeaned aquifer thickness) dramatically reduces the total variation in aquifer thickness and not enough variations are left for accurately estimating its impact of the share of irrigated acres. The inclusion of State FEs does not diminish the variation in aquifer thickness. Therefore, while the inclusion of State FEs certainly does not provide as tight of a control as county FEs, we have decided to include State FEs for both corn and soybean irrigated share regressions as it is better than not including them. Figure 7 compares the estimated impacts of aquifer thickness with or without State FEs. The negative impact of reduction in aquifer thickness on the share of irrigated acres is slightly greater when state fixed effects are included. Therefore, it may be that aquifer thickness is not very highly correlated with important unobservable variables that are affecting irrigated share significantly. Given the direction of potential bias, providing tighter controls with county fixed effects may result in even more significant drop in the share of irrigated acres (we may be underestimating the impact of aquifer thickness.). All the discussion we are providing here is now seen in Appenddix A.
}

\begin{enumerate}
\def\labelenumi{\arabic{enumi}.}
\setcounter{enumi}{3}
\tightlist
\item
  There are minor typos (e.g., missing reference listed as REF) that
  should be corrected prior to resubmission.
\end{enumerate}

\textcolor{blue}{We believe we fixed those typos.}

\end{document}
