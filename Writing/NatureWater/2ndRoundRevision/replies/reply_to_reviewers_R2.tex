% Options for packages loaded elsewhere
\PassOptionsToPackage{unicode}{hyperref}
\PassOptionsToPackage{hyphens}{url}
%
\documentclass[
]{article}
\usepackage{amsmath,amssymb}
\usepackage{lmodern}
\usepackage{url}
\usepackage{iftex}
\ifPDFTeX
  \usepackage[T1]{fontenc}
  \usepackage[utf8]{inputenc}
  \usepackage{textcomp} % provide euro and other symbols
\else % if luatex or xetex
  \usepackage{unicode-math}
  \defaultfontfeatures{Scale=MatchLowercase}
  \defaultfontfeatures[\rmfamily]{Ligatures=TeX,Scale=1}
\fi
% Use upquote if available, for straight quotes in verbatim environments
\IfFileExists{upquote.sty}{\usepackage{upquote}}{}
\IfFileExists{microtype.sty}{% use microtype if available
  \usepackage[]{microtype}
  \UseMicrotypeSet[protrusion]{basicmath} % disable protrusion for tt fonts
}{}
\makeatletter
\@ifundefined{KOMAClassName}{% if non-KOMA class
  \IfFileExists{parskip.sty}{%
    \usepackage{parskip}
  }{% else
    \setlength{\parindent}{0pt}
    \setlength{\parskip}{6pt plus 2pt minus 1pt}}
}{% if KOMA class
  \KOMAoptions{parskip=half}}
\makeatother
\usepackage{xcolor}
\usepackage[margin=1in]{geometry}
\usepackage{graphicx}
\makeatletter
\def\maxwidth{\ifdim\Gin@nat@width>\linewidth\linewidth\else\Gin@nat@width\fi}
\def\maxheight{\ifdim\Gin@nat@height>\textheight\textheight\else\Gin@nat@height\fi}
\makeatother
% Scale images if necessary, so that they will not overflow the page
% margins by default, and it is still possible to overwrite the defaults
% using explicit options in \includegraphics[width, height, ...]{}
\setkeys{Gin}{width=\maxwidth,height=\maxheight,keepaspectratio}
% Set default figure placement to htbp
\makeatletter
\def\fps@figure{htbp}
\makeatother
\setlength{\emergencystretch}{3em} % prevent overfull lines
\providecommand{\tightlist}{%
  \setlength{\itemsep}{0pt}\setlength{\parskip}{0pt}}
\setcounter{secnumdepth}{-\maxdimen} % remove section numbering
\ifLuaTeX
  \usepackage{selnolig}  % disable illegal ligatures
\fi
\IfFileExists{bookmark.sty}{\usepackage{bookmark}}{\usepackage{hyperref}}
\IfFileExists{xurl.sty}{\usepackage{xurl}}{} % add URL line breaks if available
\urlstyle{same} % disable monospaced font for URLs
\hypersetup{
  hidelinks,
  pdfcreator={LaTeX via pandoc}}

\author{}
\date{\vspace{-2.5em}}

\begin{document}

\section{Reply to reviewer \#1}

\textcolor{blue}{Thank you for your feedback on our manuscript. We have addressed all your comments and suggestions, and our replies to your comments can found in blue below (original comments in black).}

I have read the revised manuscript and the authors’ responses to my initial review. I think they have done a nice job in responding to the points I raised. Thus, I recommend acceptance after the authors address the minor points that I noted while reading the revised manuscript. Those points are as follows:

\begin{itemize}
    \item Line 58 – ``farmers'' should be ``farmers’''.

\textcolor{blue}{Thank you for catching this. Fixed}

\item Figure 1 caption – I found the term ``county-year water deficit values'' confusing. I recommend use of something like ``county-level average water deficit values''. Might also consider modifying line 290.

\textcolor{blue}{We agree. Instead of county-level "average" water deficit, we say county-level annual water deficit because they are not averaged values.}

\item Line 305 – replace ``aquifer'' with ``aquifer thickness'' or ``they''.

\textcolor{blue}{Thank you for catching this. Fixed}

\item Equations 1 and 2 – shouldn’t ``k'' be ``k=1'' – if not, please explain how k is determined. Also, explain ``K'' and how it was determined.

\textcolor{blue}{Please see line 327 for the explanation of what K represents and how K is selected. We added similar explanations for equation 2 at line 341.}

\item Lines 385-387 – Figure A.3 is not presenting results by aquifer thickness category. It is presenting results for three aquifer thicknesses, not categories. Please rewrite.

\textcolor{blue}{Figure A.3 indeed presents what we intended to present. Please note that we made similar changes for average productivity analysis in the main text as shown in Figures 4 and 5. While the irrigated yield regression (equation 1) uses aquifer thickness category, the irrigated share regression uses aquifer thickness as it is as a continuous variable. In our average productivity analysis, we use four aquifer thickness values: 10, 50, 90, and 130 meter. For example, for the analysis of 10 meter, estimated yield response function for the first category is used as it falls within the [9m, 30.1 m] range. Estimated irrigation share for 10 meter can be obtained simply by plugging 10 into the trained model of the share of irrigated acres. In our initial submission, we were doing the same kind of analysis (using one number from each aquifer thickness category), but reported by category. In the first round of revision, we decided to have two aquifer thickness values in the 3rd quantile category (90 and 130 meter). This is in response to another reviewer's point that it is not clear if indeed the impact of aquifer thickness on average productivity is non-linear as we claimed in our initial submission.}

\item Section B title – revise ``the the''.

\textcolor{blue}{Thank you for catching this. Fixed}

\item Figure B.1 – y-axis label is not correct – this should be ``Decline in Estimated….''.

\textcolor{blue}{Thank you for catching this. Fixed}

\item Lines 398-407 – this paragraph needs to be rewritten. Use of ``demeaning'' or ``demeaned'' is incorrect. My guess is that the authors mean something like ``detrending'' or ``detrended'' – please revise. Also, fix noun-verb disagreement, etc. in this paragraph.

\textcolor{blue}{We understand that demeaning sounds wrong, however, it is a term that is used specifically for the types of operations in Economics.}

\textcolor{red}{Tim, can you work on making the paragraph free of grammatical errors?}

\item Figure C.2 – ``soy'' should be ``soybean'' to be consistent with the rest of the manuscript.

\item Line 428 – is ``a linear in parameter model'' correct? Wouldn’t ``a linear parameter model'' be better?

\textcolor{blue}{We rewrote it to "linear-in-parameter" model. Linear-in-parameter model is a more explicit and correct way of saying what we people call a linear model.}

\end{itemize}

\section{Reply to reviewer \#3}

The manuscript has been considerably improved by the substantial edits in response to the first round reviewer comments. The highly relevant insights of the paper can now be more readily understood by readers. The inclusion of new Figures 2 and 5 and Appendices A and B make it easier to parse the results. Main text additions and Appendices C and D clarify the methods and result interpretation.

Most of the previous comments have been addressed. Some open points, however, still remain and a few new ones have arisen:

\subsection{MAIN COMMENTS}

\begin{itemize}
\item New comment 1 – Clarification about moderate result changes in Figures 1, 3, and 4: Could you explain in more detail how and why the results in these figures have changed? The response to my previous comment (7.b) mentions that the regressions were re-run with sand percentage, silt percentage, and water holding capacity. The variable ``X'' corresponding to these soil characteristics, however, is only included in Equation (2) (irrigated area) and not in Equation (1) (yields).

  \textcolor{blue}{As clearly stated in the manuscript, county fixed effects were included in equation 1 (reperesented by $\alpha_i$). This means that any county-specific time-invariant variables like those in $X$ will be dropped. This does not mean $X$ are not accounted for. They are accounted for as county fixed effects will capture them.}

  \begin{itemize}
  \item (1.a) Does the addition of ``X'' to Equation (2) explain the changed irrigated area results in Figure 3?

  \textcolor{blue}{Yes. Moreover, the results for the share of irrigated acre in this manuscript are different from the previous round due to the changes in the choice of the number of spline basis functions ($K$ and $L$). Previously, we were using 5 for both. However, in responding to reviewer 1's request to explain how the number of spline basis functions are determined, we realized that we were not using the optimal numbers. We tested different numbers earlier in the project and found that the results are not sensitive to the choice of the number of spline basis functions. So, we were simply using reasonable numbers back then. Even though the choice of $K$ and $L$ does not affect the results much, we now pick the numbers based on BIC.}
  
  \item (1.b) What motivates the other changes made to Equation (2)?

  \textcolor{blue}{The change made other than the ones explained above and including state-year FE is whether we allow the interactions of $WB$ and $AT$. Since we found that there is not significant interactions between them, we have decided to make the model more parsimonious and simply model the impact of $WB$ and $AT$ separately. The results are virtually the same whether we used the old and current models.}
  
  \item (1.c) What explains the change in the yield results in Figures 1 and 4?

  \textcolor{blue}{We should have mentioned this change in our previous replies. We found that we were actually not grouping observations into three groups in a way that their number of observations are the same as we claimed due to a coding error. We fixed this problem in the previous round. The changes observed in Figure 1 is due solely to this change. For Figure 3, the changes are combination of this change and also the change in the results of the share of irrigated acres as mentioned above. The previous round of Figure 3 of course is different from the one in the initial version as we added another aquifer level to make our claim of the non-linear impact of aquifer much clearer. Figure 3 in the current version is different from the one in the previous round as we changed the number of spline basis function for equation 2 as mentioned above. Overall, the observed changes in results are really minor.} 
  
  \item (1.d) Please add a short reference to the soil characteristics dataset used in Section 4.1 or 4.2 to allow readers can assess the assumptions made without checking the data availability statement.

  \textcolor{blue}{We added the following sentence at line 292.}
  
  \textcolor{blue}{Soil characteristics data (sand percentage, silt percentage, and water holding capacity) were obtained from the Soil Survey Geographic Dataset (SSURGO). Observation units of the SSURGO data are polygons. For each county, the SSURGO polygons were overlaid with the county boundary, and the area-weighted average of the soil variables.}

  \end{itemize}

\item New comment 2 – Figure 2 results below 200mm: Lines 99 states that quantiles 3 has ``statistically lower yields''. Please clarify this. From visual inspection of Figure 2, the quantile 3 yields do not seem to be significantly below the quantile 1 values (at least for corn). Could you please clarify this? If the result is not statistically significant, does it make sense to discuss it in lines 98-111? Since the paper otherwise applies statistical significance as a criterion for meaningful results, I would recommend being consistent.


    \textcolor{blue}{Thank you for noticing this. We did not realize the difference is no longer statistically significant. Following your suggestion, we simply dropped the pararaph describing the results.}

\item New comment 3 - Figure 4 category changes: Why were the aquifer thickness categories changed in Figure 4? Please also clarify how function estimates for point values of aquifer thickness (rather than value ranges) are derived from the data and explain what motivates using different categories in Figures 1 and 4.

    \textcolor{blue}{We are not using the different categories between Figures 1 and 4. While the irrigated yield regression (equation 1) uses aquifer thickness category, the irrigated share regression uses aquifer thickness as it is as a continuous variable. In our average productivity analysis which combines the two kinds of analysis, we use four aquifer thickness values: 10, 50, 90, and 130 meter. For example, for the analysis of 10 meter, the estimated yield response function to water deficit for the ``first'' category is used as it falls within the [9m, 30.1 m] range. The estimated irrigation share for 10 meter can be obtained simply by plugging 10 into the trained model of the share of irrigated acres (equation 2). In our initial submission, we were doing the same of analysis (using one number from each aquifer thickness category), but reported by category instead of reporting aquifer levels we were using for each category. In the first round of revision, we decided to have two aquifer thickness values in the 3rd quantile category (90 and 130 meter). 50 meter falls in the second category. This is in response to your point that it is not clear if indeed the impact of aquifer thickness on average productivity is non-linear as we claimed in our initial submission.}

\item New comment 4 – causality and fixed effects: The manuscript has been strengthened by the expansion of the section explaining the reasons for assuming declining aquifer thickness causally reduces the ability to use irrigation to maintain crop yields (lines 40-57). Please clarify the following:

  \begin{itemize}
  \item (4.a) Please clarify for readers if other major causal pathways might be relevant. For example: Are there relevant regulations that limit abstractions in counties where aquifers are too depleted? If so, the causality would be similar, but the implications somewhat different. If not, then stating this would be helpful so readers can rule out potentially strong alternative causal relationships.

  \textcolor{red}{Tim, I don't think this guys understood what we wrote.}
  
  \item (4.b) Relatedly, the responses to my previous comments 1.a and 7.b indicated that county-level fixed effects were not included in any regressions (only state-level). The response to reviewer \#2 and the Methods section suggest that county-level fixed effects were used in Equation 1. Please clarify the discrepancy and make sure this is clear in the paper.

  \textcolor{blue}{From the initial submission, it has been clearly stated that county fixed effects $\alpha_i$ are included in equation 1. However, we understand your confusion because we stated the following in our response.}

  \textcolor{blue}{``Other sources of bias (e.g., model misspecification error) will also almost always
exist. In our analysis, one of the main limitations is not including county fixed effects in our regression\textcolor{red}{s}, which would have controlled all the county-specific time-invariant characteristics.''}

  \textcolor{blue}{It is our mistake that we added \textcolor{red}{s} to ``regression''. As we stated above, we do include county fixed effect in equation 1. }
  
  \item (4.c) The following conclusion in Appendix C, lines 410-414 does not seem fully convincing in its present form: The fact that state-level fixed effects are not influential does not immediately imply that county-level fixed effects are ``unlikely to cause significant bias''. There could e.g. easily be relevant county-level effects that average out at the state level.

  \textcolor{blue}{We agree. It is too strong of a claim. We simply tone down our confidence.}

  \textcolor{blue}{We also changed our sentence in line .}
  
  \end{itemize}

\item New comment 5 – Figure 1:

  \begin{itemize}
   \item (5.a) The histograms are a helpful addition to Figure 1. Please consider adding the total number of observations to the caption to help readers interpret the values. I also recommend adding such histograms to Figures 3 and 4 for consistency. Please clarify why the histogram reaches further than the curves in Figure 1.B and correct this if necessary.

  \item (5.b) Lines 77-78 state ``rainfed yields of both crops decline significantly'' with reference to Figure 1. Please quantify, visualize, or otherwise provide evidence of the significance to readers, or rephrase.
  \end{itemize}

\end{itemize}

\subsection{MINOR COMMENTS}

\begin{itemize}

\item (M1) The response letter indicates that ``all [minor] points have been addressed''. Most minor comments were addressed correctly. The following minor comments from the previous round, however, have not yet been addressed:

\item (M1.a) P. 15: ``WB'' and ``ST'' [now ``AT''] should not be in italics, unless ``W'', ``B'', ``S'', and ``T'' are separate variables

\item (M1.b) Lines 325-326: ``with $\sigma_j$ represents the category specific intercept'' should be ``with $\sigma_j$ representing the category-specific intercept'' or something similar

\item (M1.c) Equation 2: What is ``$\mu_t$'' in equation 2?

\item (M2) Some additional minor points have arisen:

\item (M2.a) Line 315: ``Y'' is capitalized here, but not in equation (1)

\item (M2.b) Equation 2: Please replace the re-used symbol ``alpha'' with a different symbol to avoid confusion.

\item (M2.c) Lines 427-428: Please clarify the grammatical structure of the following sentence: ``Notice that unlike non-parametric regression, GAM is actually a linear in parameter model and estimate $\beta_k$ to fit the data best.''

\end{itemize}

\end{document}