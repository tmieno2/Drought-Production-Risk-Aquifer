% Options for packages loaded elsewhere
\PassOptionsToPackage{unicode}{hyperref}
\PassOptionsToPackage{hyphens}{url}
%
\documentclass[
]{article}
\usepackage{amsmath,amssymb}
\usepackage{lmodern}
\usepackage{url}
\usepackage{iftex}
\ifPDFTeX
  \usepackage[T1]{fontenc}
  \usepackage[utf8]{inputenc}
  \usepackage{textcomp} % provide euro and other symbols
\else % if luatex or xetex
  \usepackage{unicode-math}
  \defaultfontfeatures{Scale=MatchLowercase}
  \defaultfontfeatures[\rmfamily]{Ligatures=TeX,Scale=1}
\fi
% Use upquote if available, for straight quotes in verbatim environments
\IfFileExists{upquote.sty}{\usepackage{upquote}}{}
\IfFileExists{microtype.sty}{% use microtype if available
  \usepackage[]{microtype}
  \UseMicrotypeSet[protrusion]{basicmath} % disable protrusion for tt fonts
}{}
\makeatletter
\@ifundefined{KOMAClassName}{% if non-KOMA class
  \IfFileExists{parskip.sty}{%
    \usepackage{parskip}
  }{% else
    \setlength{\parindent}{0pt}
    \setlength{\parskip}{6pt plus 2pt minus 1pt}}
}{% if KOMA class
  \KOMAoptions{parskip=half}}
\makeatother
\usepackage{xcolor}
\usepackage[margin=1in]{geometry}
\usepackage{graphicx}
\makeatletter
\def\maxwidth{\ifdim\Gin@nat@width>\linewidth\linewidth\else\Gin@nat@width\fi}
\def\maxheight{\ifdim\Gin@nat@height>\textheight\textheight\else\Gin@nat@height\fi}
\makeatother
% Scale images if necessary, so that they will not overflow the page
% margins by default, and it is still possible to overwrite the defaults
% using explicit options in \includegraphics[width, height, ...]{}
\setkeys{Gin}{width=\maxwidth,height=\maxheight,keepaspectratio}
% Set default figure placement to htbp
\makeatletter
\def\fps@figure{htbp}
\makeatother
\setlength{\emergencystretch}{3em} % prevent overfull lines
\providecommand{\tightlist}{%
  \setlength{\itemsep}{0pt}\setlength{\parskip}{0pt}}
\setcounter{secnumdepth}{-\maxdimen} % remove section numbering
\ifLuaTeX
  \usepackage{selnolig}  % disable illegal ligatures
\fi
\IfFileExists{bookmark.sty}{\usepackage{bookmark}}{\usepackage{hyperref}}
\IfFileExists{xurl.sty}{\usepackage{xurl}}{} % add URL line breaks if available
\urlstyle{same} % disable monospaced font for URLs
\hypersetup{
  hidelinks,
  pdfcreator={LaTeX via pandoc}}

\author{}
\date{\vspace{-2.5em}}

\begin{document}

\section{Changes unrelated to reviewers' comments}

\textcolor{blue}{We made a small correction to our codes and data. In the previous round, we included counties with only irrigated acres for the irrigation share regression. For such counties, the share of irrigated acres was calculated to be 1. Therefore, we dropped them. While it resulted in some minor changes in our results from those provided in the previous round, all the qualitative results and conclusions are the same as the previous rounds.}

\section{Reply to reviewer \#1}

I have read the revised manuscript and the authors' responses to my initial review. I think they have done a nice job in responding to the points I raised. Thus, I recommend acceptance after the authors address the minor points that I noted while reading the revised manuscript. Those points are as follows:

\textcolor{blue}{Thank you for your feedback on our manuscript. We have addressed all your comments and suggestions, and our replies to your comments can be found in blue below (original comments in black). Please take a look at our comments about a small change in our codes at the top as well.}


\begin{itemize}
    \item Line 58 – ``farmers'' should be ``farmers'''.

\textcolor{blue}{Thank you for catching this. We have fixed all instances of this typo.}

\item Figure 1 caption – I found the term ``county-year water deficit values'' confusing. I recommend use of something like ``county-level average water deficit values''. Might also consider modifying line 290.

\textcolor{blue}{We agree. Instead of county-level "average" water deficit, in our revised manuscript we now say county-level annual water deficit because they are not averaged values.}

\textcolor{blue}{We also changed "evident in county-years with large water deficits" in line 290 in the previous manuscript to "evident in years with large water deficits," which is in line 275 in the current manuscript.}

\item Line 305 – replace ``aquifer'' with ``aquifer thickness'' or ``they''.

\textcolor{blue}{Thank you for catching this. This typo has been fixed.}

\item Equations 1 and 2 – shouldn't ``k'' be ``k=1'' – if not, please explain how k is determined. Also, explain ``K'' and how it was determined.

\textcolor{blue}{K represents the total number of spline basis functions, which influences the complexity of the spline shape in a GAM. A summary of this detail can be found on line 316 of our revised manuscript. We have also added similar explanations for equation 2 on line 330.}

\item Lines 385-387 – Figure A.3 is not presenting results by aquifer thickness category. It is presenting results for three aquifer thicknesses, not categories. Please rewrite.

\textcolor{blue}{While the irrigated yield regression (equation 1) uses aquifer thickness categories, aquifer thickness is specified as a continuous variable for our irrigated area share regression. In our average productivity analysis, the results of yield and area regression results are combined by assigning a specific aquifer thickness value (e.g. 10m) to the relevant category for the yield component and using the specific thickness value for the area component of the regression. This is a necessary step given the differences in the format (i.e. categorical vs continuous) of the aquifer thickness variable in each of the regression components of our combined average productivity model. In Figure A.3. and also Figure 5, the set of thickness values (10, 40, 70, and 100m - covering the first, second, third, and third saturated thickness categories, respectively) were selected to be equi-distance, so as to enable more direct exploration of whether the impact of aquifer thickness on average productivity is non-linear in response to a comment raised by one of the other reviewers. To clarify this feature of our average productivity model we have added the following sentence to line 341 of the methods of our revised manuscript:}

\textcolor{blue}{``In this combined model, aquifer thickness is a continuous variable for the irrigated area share regression component and a categorical variable for the irrigated yield regression following the format of Equations 2 and 1, respectively.''}

\item Section B title – revise ``the the''.

\textcolor{blue}{Thank you for catching this. This typo has been corrected.}

\item Figure B.1 – y-axis label is not correct – this should be ``Decline in Estimated….''.

\textcolor{blue}{Thank you for catching this. This typo has been corrected.}

\item Lines 398-407 – this paragraph needs to be rewritten. Use of ``demeaning'' or ``demeaned'' is incorrect. My guess is that the authors mean something like ``detrending'' or ``detrended'' – please revise. Also, fix noun-verb disagreement, etc. in this paragraph.

\textcolor{blue}{While we understand the confusion around the term demeaning, we feel it is important to retain this term in the manuscript as it an important and established terms in statistics and economics that denotes the centering of data by removing the mean. This is subtly different to detrending, and hence we retain demeaning as a term in our manuscript. Noun-verb disagreements in this paragraph have been corrected in this section of the manuscript.}

\item Figure C.2 – ``soy'' should be ``soybean'' to be consistent with the rest of the manuscript.

\textcolor{blue}{Thank you for catching this. The text has been modified for consistency as suggested.}

\item Line 428 – is ``a linear in parameter model'' correct? Wouldn't ``a linear parameter model'' be better?

\textcolor{blue}{From a statistical perspective, a ``linear'' model means that the model is linear in parameter, but not necessarily with regard to the independent variable. For example, the hypothetical model below is linear in parameter, but not linear in $x$.} 

\begin{align}
  y = \beta x^2 + v
\end{align}  

\textcolor{blue}{GAM is an example of a type of model that is linear in parameter, but non-linear with respect to their independent variable(s). We have therefore retained the term linear in parameter to ensure a clear distinction with non-linear regression models. To more clearly clarify this point, we have also rewritten the sentence flagged by the reviewer as follows:}

\textcolor{blue}{``Unlike non-parametric regression, a GAM is a linear (in-parameter) model and estimates $\beta_k$ to best fit the observation data.''}

\end{itemize}

\section{Reply to reviewer \#3}

The manuscript has been considerably improved by the substantial edits in response to the first round reviewer comments. The highly relevant insights of the paper can now be more readily understood by readers. The inclusion of new Figures 2 and 5 and Appendices A and B make it easier to parse the results. Main text additions and Appendices C and D clarify the methods and result interpretation.

Most of the previous comments have been addressed. Some open points, however, still remain and a few new ones have arisen:

\textcolor{blue}{Thank you for your feedback on our manuscript. We have addressed all your comments and suggestions, and our replies to your comments can be found in blue below (original comments in black). Please also note our comments about a small change in our codes at the top as well.}

\subsection{MAIN COMMENTS}

\begin{itemize}
\item New comment 1 – Clarification about moderate result changes in Figures 1, 3, and 4: Could you explain in more detail how and why the results in these figures have changed? The response to my previous comment (7.b) mentions that the regressions were re-run with sand percentage, silt percentage, and water holding capacity. The variable ``X'' corresponding to these soil characteristics, however, is only included in Equation (2) (irrigated area) and not in Equation (1) (yields).

  \textcolor{blue}{As stated in the manuscript, county fixed effects were included in equation 1 (represented by $\alpha_i$). This means that any county-specific time-invariant variables - such as those related to soil properties as represented in $X$ - will already be accounted for implicitly in the county-level fixed effects term. In contrast, soil property variables are included explicitly in the irrigated area share regression as this regression only includes state-year fixed effects for reasons discussed in Appendix C}

  \begin{itemize}
  \item (1.a) Does the addition of ``X'' to Equation (2) explain the changed irrigated area results in Figure 3?

  \textcolor{blue}{Yes. Please note that the results for the share of irrigated acre in this manuscript are also different from the previous version of our manuscript  due to the code updates that we have summarized at the top of this document. All the qualitative results and conclusions regarding the impacts of aquifer thickness are the same as the previous versions of our manuscript.}
  
  \item (1.b) What motivates the other changes made to Equation (2)?

  \textcolor{blue}{The other changes made to Equation 2 include the addition of state-year FE, in order to provide stronger controls than just including year FE. In addition, we choose not to allow the interactions of $WB$ and $AT$. This decision was made in the interests of developing a more parsimonious model that considers the impacts of $WB$ and $AT$ individually. Our results remain virtually identical though whether this the interaction between $WB$ and $AT$ is included or not. }
  
  \item (1.c) What explains the change in the yield results in Figures 1 and 4?

  \textcolor{blue}{Thank you for highlighting this important omission from our previous response in the last review round. In our earlier revision of the manuscript, we found a small coding error that meant we were actually not grouping observations equally into three groups as stated in our methods. This coding error was corrected in our previous revised submission, and the changes observed in Figure 1 are due solely to this change. For Figure 4 of the previous version (1st round revision), the changes are combination of this change and also the change in the results of the regression for share of irrigated acres as mentioned above. This is because Figure 4 presents the results of the analysis that combines what are presented in Figures 1 and Figure 3 of the previous version (1st round revision). Figure 3 in the 1st round and Figure 4 in the previous round are different because we added another aquifer level to make our claim of the non-linear impact of aquifer much clearer in response to one of your comments in the previous round. Further, Figure 4 in the current version is different from the one in the previous round as we changed the aquifer levels for this analysis. Instead of 10, 50, 90, and 130 meters, we use 10, 40, 70, and 100 meters. We made this change to illustrate that significant non-linear changes in total productivity can happen within a narrower range of evenly spaced aquifer thickness compared to the previous sets of aquifer thickness values used in this figure.} 
  
  \item (1.d) Please add a short reference to the soil characteristics dataset used in Section 4.1 or 4.2 to allow readers can assess the assumptions made without checking the data availability statement.

  \textcolor{blue}{We added the following sentence at line 277.}
  
  \textcolor{blue}{Soil characteristics data (sand percentage, silt percentage, and water holding capacity) were obtained from the Soil Survey Geographic Dataset (SSURGO). Observation units of the SSURGO data are polygons. For each county, the SSURGO polygons were overlaid with the county boundary, and the area-weighted average of the soil variables were calculated.}

  \end{itemize}

\item New comment 2 – Figure 2 results below 200mm: Lines 99 states that quantiles 3 has ``statistically lower yields''. Please clarify this. From visual inspection of Figure 2, the quantile 3 yields do not seem to be significantly below the quantile 1 values (at least for corn). Could you please clarify this? If the result is not statistically significant, does it make sense to discuss it in lines 98-111? Since the paper otherwise applies statistical significance as a criterion for meaningful results, I would recommend being consistent.


    \textcolor{blue}{Thank you for noticing this. You are correct that this difference is no longer statistically significant. Following your suggestion, we have removed this paragraph from the results to ensure consistency in what is discussed as significant in the context of our analysis.}

\item New comment 3 - Figure 4 category changes: Why were the aquifer thickness categories changed in Figure 4? Please also clarify how function estimates for point values of aquifer thickness (rather than value ranges) are derived from the data and explain what motivates using different categories in Figures 1 and 4.


    \textcolor{blue}{Thank you for this helpful point, which highlgihts a need to clarify some details about how aquifer thickness enters different parts of our analysis. While the irrigated yield regression (equation 1, Figure 1) uses aquifer thickness category, the irrigated share regression (Figure 4) uses aquifer thickness as it is as a continuous variable. In our average productivity analysis (Figure 4) which combines these two regressions, aquifer thickness is specified as a specific point value for the component related to irrigated area share and as a categorical value for component related to irrigated per-area yields. This step has been the same throughout all versions of our manuscript, and is not a change that has been introduced in current or prior revisions. We have added the following sentence on line 341 of our methods to clarify this point:}

    \textcolor{blue}{``In this combined model, aquifer thickness is a continuous variable for the irrigated area share regression component and a categorical variable for the irrigated yield regression following the format of Equations 2 and 1, respectively.''}
    
    \textcolor{blue}{In the previous version of our average productivity analysis, we assess the impacts of four aquifer thickness values: 10, 50, 90, and 130 meters in the initial version of our manuscript, with equal spacing chosen to illustrate non-linearity in impacts of aquifer thickness. In our last round of revisions, we changed these values to instead be 10, 40, 70, and 100m. These values ensure that we have a range of values spanning different aquifer thickness categories. Please note that key qualitative message and findings is identical whichever set of values we use. As we mentioned earlier, we made this change to illustrate that significant non-linear changes in total productivity can happen within a narrower range of evenly spaced aquifer thickness compared to the previous sets of aquifer thickness values used in this figure. This change also provide results that are more realistically centered on the range of aquifer thickness values observed within our study area and period.} 

\item New comment 4 – causality and fixed effects: The manuscript has been strengthened by the expansion of the section explaining the reasons for assuming declining aquifer thickness causally reduces the ability to use irrigation to maintain crop yields (lines 40-57). Please clarify the following:

  \begin{itemize}
  \item (4.a) Please clarify for readers if other major causal pathways might be relevant. For example: Are there relevant regulations that limit abstractions in counties where aquifers are too depleted? If so, the causality would be similar, but the implications somewhat different. If not, then stating this would be helpful so readers can rule out potentially strong alternative causal relationships.


  \textcolor{blue}{Regulatory pumping restrictions have limited impacts on water use decisions in our study area, and therefore are unlikely to impact the ability to use groundwater as a buffer against drought. In contrast, the impacts of aquifer depletion on farmers' ability to meet crop water needs through changes in well yields are linked directly to hydrologic and agronomic theory, and have demonstrated extensively through prior work cited in our manuscript. While we note that restrictions in pumping are imposed in some parts of the High Plains Aquifer, there are several reasons why we can confidently state these would not be a confounding source of causality in our analysis. Please see the list below. We added the following paragraphs to a new Appendix A in this round of manuscript, as we feel these are important points to include but are constrained by word limits in the main manuscript.}

  \begin{enumerate}
      \item \textcolor{blue}{The majority of groundwater withdrawal regulations that do exist in the HPA limit pumping to a specified volume over a multiyear period, and farmers have also been allowed to bank and carryover significant volumes of water deficits from historic periods when allocation levels were higher (Rimsaite and Brozovic, 2023). This enables farmers to increase water withdrawals substantially in drought years when demand is higher, and prior research has shown that farmers' water use decisions are not constrained by such regulations even in extreme drought years such as 2012 (Foster et al., 2019).}
      \item \textcolor{blue}{Where more restrictive groundwater withdrawal limits have been introduced, for example in some Groundwater Management Districts in Kansas (Marston et al., 2022; Rimsaite and Brozoic, 2023), these have universally been implemented in the last 5-10 years and therefore have no or only minimal overlap with our study period which extends up to 2016. The earliest of these groundwater managements in Kansas interventions was the Sheridan 6 Local Enhanced Management Area (LEMA) in NW Kansas, which introduced restrictions to farmer groundwater pumping of 11 inches per year on average per 5-year period starting in 2013. While this overlaps with the final 4 years of our study period, the LEMA area covers only parts of two counties and hence represents a very small proportion of our overall study sample while the years 2013-2016 were all characterized by average or above average growing season rainfall (i.e. not drought years). Moreover, evidence shows that the effect of restrictions introduced by the LEMA has primarily been a reduction in per-area water use, with irrigated crop yields largely unaffected and only small reductions in irrigated areas compared to pre-policy conditions (Drysdale and Hendricks, 2018; Deines et al., 2021). These policies therefore would not be expected to confound our results on the impacts of aquifer thickness.}
      \item \textcolor{blue}{Areas of the southern HPA, in particular in northern Texas, where the lowest aquifer thickness values are observed have few if any regulatory restrictions on groundwater extraction due to laws that typically provide farmers with absolute rights to use groundwater under their land coupled with low levels of metering that constrain potential for implementing extensive and binding pumping restrictions (Wheeler et al., 2016; Closas and Molle, 2018).}
  \end{enumerate}

  \textcolor{blue}{References referred to above are as follows:
  \begin{itemize}
      \item Rimsaite, R., and Brozovic, N.  (2023). Groundwater Development Paths in the U.S. High Plains. Oxford Research Encyclopedia of Environmental Science. 
      \item Foster, T., Goncalves, I. Z., Campos, I., Neale, C. M. U., and Brozovic, N. (2019). Assessing landscape scale heterogeneity in irrigation water use with remote sensing and in situ monitoring. Environmental Research Letters, 14(2), 024004.
      \item Marston, L. T., Zipper, S., Smith, S. M., Allen, J. J., Butler, J. J., Gautam, S., and David, J. Y. (2022). The importance of fit in groundwater self-governance. Environmental Research Letters, 17(11), 111001.
      \item Closas, A., and Molle, F. (2018). Chronicle of a Demise Foretold: State vs. Local Groundwater Management in Texas and the High Plains Aquifer System. Water Alternatives, 11(3), 511-532.
      \item Wheeler, S. A., Schoengold, K., and Bjornlund, H. (2016). Lessons to be learned from groundwater trading in Australia and the United States. Integrated Groundwater Management: Concepts, Approaches and Challenges, 493-517.
      \item Deines, J. M., Kendall, A. D., Butler, J. J., Basso, B., and Hyndman, D. W. (2021). Combining remote sensing and crop models to assess the sustainability of stakeholder‐driven groundwater management in the US high plains aquifer. Water Resources Research, 57(3), e2020WR027756.
      \item Drysdale, K. M., and Hendricks, N. P. (2018). Adaptation to an irrigation water restriction imposed through local governance. Journal of Environmental Economics and Management, 91, 150-165.
  \end{itemize}
  }
  
  \item (4.b) Relatedly, the responses to my previous comments 1.a and 7.b indicated that county-level fixed effects were not included in any regressions (only state-level). The response to reviewer \#2 and the Methods section suggest that county-level fixed effects were used in Equation 1. Please clarify the discrepancy and make sure this is clear in the paper.

  \textcolor{blue}{In our initial submission, we noted in our methods that county fixed effects $\alpha_i$ were included in equation 1 but not in equation 2. The reason for this difference is outlined in Appendix C, which provides a detailed discussion of why including county fixed effects for our irrigated area share regression is problematic.}
  
  \item (4.c) The following conclusion in Appendix C, lines 410-414 does not seem fully convincing in its present form: The fact that state-level fixed effects are not influential does not immediately imply that county-level fixed effects are ``unlikely to cause significant bias''. There could e.g. easily be relevant county-level effects that average out at the state level.

  \textcolor{blue}{We recognize your point that this may be potentially too strong of a claim. To more carefully summarise the implications of our choice of fixed effects, we have modified the text on lines 448-453 for our revision of the statement.}

  \textcolor{blue}{"This indicates that, even though there may be unobservables that are spatially correlated with aquifer thickness beyond soil characteristics, their correlation may not cause significant bias to our estimation and findings. We note that more detailed assessment of the impacts of including county-fixed effects is beyond the scope of our available datasets, and we defer this matter to future studies where greater temporal variations in aquifer thickness levels may be available to allow incorporation of county-level fixed effects."}

  \textcolor{blue}{We also removed the sentence below that was on lines 345 - 346 in the previous version of our manuscript to reflect that we can only hypothesize about the likely differences between using state and county fixed effects:}

  \textcolor{blue}{"Instead, through robustness checks, we demonstrate that our results are consistent when using state fixed effects."}

  \end{itemize}

\item New comment 5 – Figure 1:

  \begin{itemize}
   \item (5.a) The histograms are a helpful addition to Figure 1. Please consider adding the total number of observations to the caption to help readers interpret the values. I also recommend adding such histograms to Figures 3 and 4 for consistency. Please clarify why the histogram reaches further than the curves in Figure 1.B and correct this if necessary.

   \textcolor{blue}{We have added total number of observations to the captions of Figures 1 and 3. Figure 4 is the results of simulation based on the estimated model represented by equations 1 and 2, and therefore there is no appropriate number of observations for the figure so the value is omitted.}

   \textcolor{blue}{We have added histograms for Figures 3 and 4. The inconsistency in the range of the histograms and the main figures were caused by a small coding error that did not correctly align two figures vertically. These problems are now fixed. Please note that because of the width (range) of the histogram bins, it looks like the bins at the left and right ends are beyond the range of the main figure, but this is not the case and is simply a plotting feature.}

  \item (5.b) Lines 77-78 state ``rainfed yields of both crops decline significantly'' with reference to Figure 1. Please quantify, visualize, or otherwise provide evidence of the significance to readers, or rephrase.

  \textcolor{blue}{Figure A.1 provides the estimated response curve for rainfed production. As is visible, the confidence intervals are narrow at all the water deficit values. We do not think we need additional figures or explanations are necessarily for both statistical and real-world significance of the impacts of water deficit on rainfed production.}

  \end{itemize}

\end{itemize}

\subsection{MINOR COMMENTS}

\begin{itemize}

\item (M1) The response letter indicates that ``all [minor] points have been addressed''. Most minor comments were addressed correctly. The following minor comments from the previous round, however, have not yet been addressed:

\textcolor{blue}{We apologize for this mistake. This time, all the minor comments are addressed.}

\item (M1.a) P. 15: ``WB'' and ``ST'' [now ``AT''] should not be in italics, unless ``W'', ``B'', ``S'', and ``T'' are separate variables

\textcolor{blue}{Instead of making $WB$ and $AT$ un-italicized, we simply replaced them with $D$ and $A$, respectively for all of their instances.}

\item (M1.b) Lines 325-326: ``with $\sigma_j$ represents the category specific intercept'' should be ``with $\sigma_j$ representing the category-specific intercept'' or something similar

\textcolor{blue}{You are correct. This has been corrected in the revised manuscript.}

\item (M1.c) Equation 2: What is ``$\mu_t$'' in equation 2?

\textcolor{blue}{``$\mu_t$'' represents year fixed effects. However, as we now use state-year FE represented by $\theta_{s,t}$, the ``$\mu_t$' should have been removed. Equation 2 no longer includes $\mu_t$.}

\item (M2) Some additional minor points have arisen:

\item (M2.a) Line 315: ``Y'' is capitalized here, but not in equation (1)

\textcolor{blue}{We changed $y_{i,t}$ in equation (1) to $Y_{i,t}$.}

\item (M2.b) Equation 2: Please replace the re-used symbol ``alpha'' with a different symbol to avoid confusion.

\textcolor{blue}{We changed all the occurrences of $\alpha_{k}$ to $\eta_{k}$.}

\item (M2.c) Lines 427-428: Please clarify the grammatical structure of the following sentence: ``Notice that unlike non-parametric regression, GAM is actually a linear in parameter model and estimate $\beta_k$ to fit the data best.''

\textcolor{blue}{We have rewritten the sentence as follows:}

\textcolor{blue}{``Unlike non-parametric regression, a GAM is a linear (in-parameter) model and estimates $\beta_k$ to best fit the observation data.''}

\end{itemize}

\end{document}